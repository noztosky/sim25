{\def\LTcaptype{none} % do not increment counter
\begin{longtable}[]{@{}
  >{\raggedright\arraybackslash}p{(\linewidth - 0\tabcolsep) * \real{1.0000}}@{}}
\toprule\noalign{}
\begin{minipage}[b]{\linewidth}\raggedright
\textbf{항공우주시스템공학회 2024년도 춘계학술대회 논문작성법}

홍길동\textsuperscript{1,†}· 이순신\textsuperscript{2}· 장보고\textsuperscript{2}

\textsuperscript{1}항공우주시스템공학회 사무국

\textsuperscript{2}항공우주시스템공학회 편집부

How to Prepare the Manuscript for SASE 2024 Spring Conference

Gildong Hong\textsuperscript{1,†}, Soonshin Lee\textsuperscript{2} and Bohgoh Jang\textsuperscript{2}

\textsuperscript{1}Secretariat of SASE

\textsuperscript{2}Editorial department of SASE
\end{minipage} \\
\midrule\noalign{}
\endhead
\bottomrule\noalign{}
\endlastfoot
\textbf{Abstract} : 항공우주시스템공학회 학술대회 양식 중 초록의 글자제한은 400자 이내입니다. 항공우주시스템공학회 학술대회 양식 중 초록의 글자제한은 400자 이내입니다. 항공우주시스템공학회 학술대회 양식 중 초록의 글자제한은 400자 이내입니다. 항공우주시스템공학회 학술대회 양식 중 초록의 글자제한은 400자 이내입니다. 항공우주시스템공학회 학술대회 양식 중 초록의 글자제한은 400자 이내입니다. 항공우주시스템공학회 학술대회 양식 중 초록의 글자제한은 400자 이내입니다. 항공우주시스템공학회 학술대회 양식 중 초록의 글자제한은 400자 이내입니다. 항공우주시스템공학회 학술대회 양식 중 초록의 글자제한은 400자 이내입니다. \\
\textbf{Key Words} : Staging System(단분리 시스템), Staging System(단분리 시스템), Staging System(단분리 시스템), Staging System(단분리 시스템), Staging System(단분리 시스템) \\
\end{longtable}
}

1. 서 론

아래에 보이는 `저자 소속'은 내용만 바꾸어서 편집하시면 됩니다{[}1{]}. 참고문헌은 첫 무장의 끝부분과 같이 bracket `{[} {]}'을 사용합니다. 아래에 보이는 저자 소속은 내용만 바꾸어서 편집하시면 됩니다. 아래에 보이는 저자 소속은 내용만 바꾸어서 편집하시면 됩니다. 아래에 보이는 저자 소속은 내용만 바꾸어서 편집하시면 됩니다. 아래에 보이는 저자 소속은 내용만 바꾸어서 편집하시면 됩니다. 아래에 보이는 저자 소속은 내용만 바꾸어서 편집하시면 됩니다. 아래에 보이는 저자 소속은 내용만 바꾸어서 편집하시면 됩니다.

{\def\LTcaptype{none} % do not increment counter
\begin{longtable}[]{@{}
  >{\raggedright\arraybackslash}p{(\linewidth - 0\tabcolsep) * \real{0.4118}}@{}}
\toprule\noalign{}
\begin{minipage}[b]{\linewidth}\raggedright
†교신저자 ( Corresponding Author )

E-mail: ABCD@sase.or.kr

Copyright Ⓒ The Society for Aerospace System Engineering
\end{minipage} \\
\midrule\noalign{}
\endhead
\bottomrule\noalign{}
\endlastfoot
\end{longtable}
}

2. 그림과 관련된 편집요령

2.1 문장에서 그림(Fig. 1)을 지칭하는 방법

이제 다음에는 문장에서 Fig. 1을 지칭하는 방법을 설명한다.

Figure 1에서와 같이 문단의 첫 단어를 Figure로 시작할 때는 Fig. 1이라고 쓰지 않고 Figure 1으로 쓴다. Fig. 1과 같이 문단의 중간에서 새로 시작하는 문장의 첫 단어는 Figure 1이 아닌 Fig. 1이라고 약자로 쓴다. 당연하게도 문장의 모든 중간에서는 Fig. 1이라고 써야 할 것이다.

이 규정은 Equation의 인용에서도 동일하게 적용된다. 문장의 중간, 혹은 문단의 첫 문장이 아닌 한, Eq. 3처럼 적어야 한다.

2.2 그림 제목(Figure Caption)

그림 제목은 Fig. 1에서 보이는 바와 같이 아래아한글의 그림캡션 기능을 이용하여 그림의 아래쪽에 작성하고, 반드시 영문으로 기술한다. 캡션의 길이가 한 줄을 넘지 않을 때는 가운데정렬하고, 한 줄을 넘을 경우에는 먼저 양쪽정렬을 한 후에 Fig. 2 다음에 오는 첫 글자 앞에 커서를 놓고 Shift+Tab을 하여 내어쓰기 정렬한다. Fig. 1, Fig. 2와 같이 그림번호는 \textbf{진하게} 쓰기를 한다.

그림의 개체속성에서 캡션과 그림 개체와의 간격은 3 mm가 적당하다.

\includegraphics[width=1.9685in,height=1.94522in,alt={C:\textbackslash Users\textbackslash User\textbackslash Desktop\textbackslash 항공우주시스템공학회 로고.bmp}]{media/image1.png}

2.3 그림의 테두리

그림에는 테두리가 없어야 한다.

3. 표와 관련된 편집요령

3.1 표 제목 (Table Caption)

표 제목 또한 Table 1에서 보이는 바와 같이 아래아한글의 캡션 기능을 이용하되 그림과 달리 표의 윗쪽에 영문으로 작성한다. 캡션의 길이가 한 줄을 넘지 않을 때는 가운데정렬하고, 한 줄을 넘을 경우에는 먼저 양쪽정렬을 한 후에 Table 2 다음에 오는 첫 글자 앞에 커서를 놓고 Shift+Tab을 하여 내어쓰기 정렬한다.

표의 개체속성에서 캡션과 표 개체와의 간격은 3 mm가 적당하다.

\begin{center}\rule{0.5\linewidth}{0.5pt}\end{center}

{\def\LTcaptype{none} % do not increment counter
\begin{longtable}[]{@{}
  >{\centering\arraybackslash}p{(\linewidth - 10\tabcolsep) * \real{0.0851}}
  >{\centering\arraybackslash}p{(\linewidth - 10\tabcolsep) * \real{0.0688}}
  >{\centering\arraybackslash}p{(\linewidth - 10\tabcolsep) * \real{0.0673}}
  >{\centering\arraybackslash}p{(\linewidth - 10\tabcolsep) * \real{0.0772}}
  >{\centering\arraybackslash}p{(\linewidth - 10\tabcolsep) * \real{0.0960}}
  >{\centering\arraybackslash}p{(\linewidth - 10\tabcolsep) * \real{0.0726}}@{}}
\toprule\noalign{}
\begin{minipage}[b]{\linewidth}\centering
Test
\end{minipage} & \begin{minipage}[b]{\linewidth}\centering
A
\end{minipage} & \begin{minipage}[b]{\linewidth}\centering
B\textsuperscript{*}
\end{minipage} & \begin{minipage}[b]{\linewidth}\centering
C
\end{minipage} & \begin{minipage}[b]{\linewidth}\centering
D
\end{minipage} & \begin{minipage}[b]{\linewidth}\centering
E
\end{minipage} \\
\midrule\noalign{}
\endhead
\bottomrule\noalign{}
\endlastfoot
Burn-in & 0 & 0.1 & 1.0 & 1 & 5 \\
1 & 0 & 0 & S/S\textsuperscript{**} & 0 & 1 \\
\multicolumn{6}{@{}>{\centering\arraybackslash}p{(\linewidth - 10\tabcolsep) * \real{0.4670} + 10\tabcolsep}@{}}{%
*Test0, **Test1} \\
\end{longtable}
}

기타 주의해야 할 편집요령은 다음과 같다.

(1) 숫자와 단위사이에는 space를 둔다. 단, \%는 붙인다.

4. 참고문헌의 인용

참고문헌은 반드시 본문에서 언급, 혹은 인용되어야 목록에 나타날 수 있다{[}2, 3{]}. 참고문헌은 다음에 오는 `참고문헌'의 예를 따라서 기술한다. Format은 15 pt 내어쓰기와 15 pt Tab을 사용하여 번호뒤의 첫글자 위치를 정렬한다{[}4-10{]}.

참고문헌의 구성은 정기간행지의 경우는 저자, 제목, 지명, 권호, 페이지, 발간연도 순으로 작성하고, 제목은 큰 따옴표 표시를 한다. 단행본의 경우는 저자, 서명, 권, 출판사명, 출판사, 소재지, 발간 연도, 페이지 순으로 표시하며, 서명은 기울여 쓴다.

항공우주시스템공학회 학술대회 양식 중 초록의 글자제한은 400자 이내입니다.

참 고 문 헌

{[}1{]} S. Arimoto, ``Linear controllable systems,'' \emph{Nature}, vol. 135, pp. 18-27, July 1990.

{[}2{]} R. C. Baker and B. Charlie, ``Nonlinear unstable systems,'' \emph{International Journal of Control}, vol. 23, no. 4, pp. 123-145, May 1989.

{[}3{]} G. S. Choi and C. S. Kim, ``Linear stable systems,'' \emph{IEEE Trans. of Automatic Control}, vol. 33, no. 3, pp. 1234-1245, Dec. 1993.

{[}4{]} M. Young, \emph{The Technical Writer\textquotesingle s Handbook}, 3rd Ed., Mill Valley, Seoul, 1989.
