\documentclass[a4paper,10pt,twocolumn]{article}

% XeLaTeX 한글 설정
\usepackage{fontspec}
\usepackage{kotex}
\setmainfont{Times New Roman}
\setsansfont{Arial}
\setmonofont{Courier New}
% 한글 글꼴 크기를 라틴과 맞추기 위해 스케일 조정
% 필요 시 0.88~1.00 사이로 미세 조정하세요
\setmainhangulfont[Scale=0.92]{Noto Serif CJK KR}
\setsanshangulfont[Scale=0.92]{Noto Sans CJK KR}

% 일반 패키지
\usepackage{graphicx}
\usepackage{booktabs}
\usepackage{siunitx}
\usepackage{hyperref}
\usepackage{tikz}
\usetikzlibrary{positioning,arrows.meta,fit,calc,matrix}
\usepackage{titling} % 제목 상단 여백 제어
\usepackage{titlesec} % 섹션 간격 제어
\usepackage{enumitem} % 리스트 간격 제어
\usepackage{caption} % 캡션 간격/서식 제어
\usepackage{listings} % 코드 블록(자동 줄바꿈)
\usepackage{float} % [H] 고정 배치를 위한 패키지
\usepackage{newunicodechar} % 유니코드 하이픈 대체
\usepackage{etoolbox} % 환경 진입 훅(표 전역 글자 크기 설정)
% 여백/단폭 설정
\usepackage[a4paper,top=22mm,bottom=22mm,left=18mm,right=18mm]{geometry}
\setlength{\columnsep}{8mm}
% siunitx v3: replace deprecated detect-all
\sisetup{
  mode = match,
  propagate-math-font = true,
  reset-math-version = false,
  reset-text-family = false,
  reset-text-series = false,
  reset-text-shape = false,
  text-family-to-math = true,
  text-series-to-math = true
}

% 제목 상단 여백 줄이기 (필요 시 값 미세 조정)
\setlength{\droptitle}{-10mm}

% 섹션/소섹션 위아래 간격 축소 (before, after)
\titlespacing*{\section}{0pt}{0.8ex plus .2ex minus .1ex}{1.2ex}
\titlespacing*{\subsection}{0pt}{0.6ex plus .1ex minus .1ex}{0.4ex}

% 문단 간격/들여쓰기 조정 (촘촘한 본문)
\setlength{\parskip}{0.25ex plus 0.1ex minus 0.1ex}
\setlength{\parindent}{1.4em}

% 특수 하이픈(비분리 하이픈 U+2011 등)을 일반 하이픈으로 치환
% Map non-breaking hyphen(U+2011) and en dash(U+2013).
% Do NOT map ASCII '-' (causes newunicodechar error).
\newunicodechar{‑}{-} % U+2011 non-breaking hyphen
\newunicodechar{–}{-} % U+2013 (en dash -> ASCII hyphen)

% 제목 크기(폰트) 축소: 섹션/소섹션을 본문 크기로 맞춤
\titleformat{\section}{\bfseries\large}{\thesection.}{0.5em}{}
\titleformat{\subsection}{\bfseries\normalsize}{\thesubsection.}{0.5em}{}
\titleformat{\subsubsection}{\bfseries\small}{\thesubsubsection.}{0.5em}{}

% 리스트(항목) 간격 축소
\setlist[itemize]{noitemsep, topsep=0.4ex, leftmargin=1.4em}
\setlist[enumerate]{noitemsep, topsep=0.4ex, leftmargin=1.4em}
% 그림 캡션과 본문(그림) 간격 축소
\captionsetup[figure]{skip=2pt}
% 리스트 코드 캡션 좌측 정렬
\captionsetup[lstlisting]{singlelinecheck=false, justification=raggedright}
% 코드 기본 설정: 작은 폰트, 자동 줄바꿈, 고정폭, 여백 최소화
\lstset{
  basicstyle=\scriptsize\ttfamily,
  breaklines=true,
  breakatwhitespace=false,
  columns=fullflexible,
  keepspaces=true,
  frame=single,
  xleftmargin=0pt,
  xrightmargin=0pt
}

% 모든 표(table) 환경의 글자 크기를 일괄 더 축소(통일)
\AtBeginEnvironment{table}{\tiny\setlength{\tabcolsep}{1.5pt}\renewcommand{\arraystretch}{0.90}}

% 간단 키워드 매크로
\newcommand{\keywords}[1]{\par\noindent\textbf{키워드—} #1}

% JASE 스타일: 제목/저자/소속/라벨 글자 크기 통일
% - 한글 제목: 14/16pt, 영문 제목: 12/14pt
% - 저자: normalsize(10pt), 소속: footnotesize(8pt)
% - 라벨(Abstract/초록/Key Words): 굵은 10pt
\newcommand{\JASEKTitle}[1]{{\bfseries\fontsize{16}{18}\selectfont #1}}
\newcommand{\JASEETitle}[1]{{\bfseries\fontsize{14}{16}\selectfont #1}}
\newcommand{\JASEAuthors}[1]{{\fontsize{10}{12}\selectfont #1}}
\newcommand{\JASEAffil}[1]{{\fontsize{8}{9.5}\selectfont #1}}
\newcommand{\JASELabel}[1]{{\bfseries\fontsize{10}{12}\selectfont #1}}
\newcommand{\JASECenterLabel}[1]{{\centering \JASELabel{#1} \par}}
% 교신저자 박스 매크로
\newcommand{\JASECorrBox}[1]{\noindent\fbox{\begin{minipage}{0.42\columnwidth}\footnotesize
\textdagger\ 교신저자 (Corresponding Author)\\[2pt]
E-mail: #1\\[2pt]
Copyright \textcopyright\ The Society for Aerospace System Engineering
\end{minipage}}}

\title{AirSim 연동 고주파 제어 벤치마크·안정화 툴킷\\
AirSim-Integrated High-Frequency Control Benchmark · Stabilization Toolkit}

\author{김경목*\,\, 이남훈**\,\, 김성권**}
\date{}

\begin{document}
% JASE-style title/abstract block spanning both columns
\twocolumn[
\vspace*{-4mm}
\noindent\makebox[0.5\textwidth][l]{\scriptsize }% left empty slot
\makebox[0.5\textwidth][r]{\scriptsize EISSN 2508-7150}\\
\noindent\makebox[0.5\textwidth][l]{\footnotesize\itshape Journal of Aerospace System Engineering}%
\makebox[0.5\textwidth][r]{\footnotesize http://dx.doi.org/10.20910/JASE/\#\#\#\#.\#\#.\#\#}\\
\rule{\textwidth}{0.4pt}
\vspace{\baselineskip}
{\centering
\JASEKTitle{AirSim 연동 고주파 제어 벤치마크·안정화 툴킷}\par\vspace{2pt}
\JASEAuthors{김경목$^{1\dagger}$\,\, 이남훈$^{2}$\,\, 김성권$^{2}$}\par\vspace{2pt}
\JASEAffil{$^{1}$교신저자 소속(예시)}\par
\JASEAffil{$^{2}$공동저자 소속(예시)}\par\vspace{4pt}
\JASEETitle{AirSim-Integrated High-Frequency Control Benchmark · Stabilization Toolkit}\par\vspace{2pt}
}
\vspace{-1mm}
\noindent\rule{\textwidth}{0.4pt}
\JASECenterLabel{Abstract}
We present a shared-memory pipeline between AirSim and an external Virtual Flight Controller (FC) that enables high-rate sensing and control with direct transfer to onboard FCs. AirSim publishes IMU 1000 Hz, barometer 100 Hz, magnetometer 100 Hz, and GNSS position 10 Hz into an in-process ring buffer; the Virtual FC consumes these streams, runs an attitude P and rate PI+D controller, and returns PWM at 400 Hz through the same shared-memory path. Compared to RPC, the pipeline reduces latency and jitter and sustains the target rates. In experiments with injected motor/gyro biases and dynamic accelerations, the system reduces peak-to-peak spikes and saturation occupancy, achieves yaw stop (rate→0), and improves attitude RMS. The architecture mirrors FC software structure, allowing methods validated in simulation to be moved to the flight controller with minimal changes. This work contributes a practical, high-frequency AirSim–FC loop and a stabilization recipe that is immediately useful for simulator and HIL development.\par\vspace{4pt}
\JASECenterLabel{초록}
AirSim–Virtual FC 간 공유메모리 파이프라인으로 고율 센싱·제어와 FC 직이식을 달성한다. AirSim은 IMU 1000 Hz, Baro 100 Hz, Mag 100 Hz, GNSS 10 Hz를 게시하고, Virtual FC는 이를 소비해 자세 P–각속도 PI+D를 수행한 뒤 PWM 400 Hz를 반환한다. RPC 대비 지연·지터를 낮추고 목표 주파수를 안정 유지한다. 모터/자이로 바이어스와 동적 가속 조건에서 피크‑투‑피크·포화 점유율이 감소하고 yaw 정지(rate→0), 자세 RMS가 개선됐다. 구조가 실제 FC와 동일해 시뮬레이터 검증을 최소 변경으로 기체에 이식 가능하다.\par\vspace{2pt}
\noindent\JASELabel{Key Words}: Multirotor, attitude control, D-term filtering, anti-windup, estimator smoothing, shared-memory IPC, AirSim
\par\vspace{2pt}\noindent\rule{\textwidth}{0.4pt}
% 교신저자 박스 (첫 페이지 상단 영역에서 작게 표시)
\vspace{2pt}
\JASECorrBox{ABCD@sase.or.kr}
\vspace{4mm}
]

\section{서론}
소형 멀티로터의 자세 제어는 모터/기체 비대칭, 센서 바이어스와 고주파 노이즈, 추정기 게인의 상황별 스위칭 등 실환경 요인으로 인해 출력 스파이크(“팅김”)와 장기적 yaw 드리프트가 빈번하게 발생한다. 본 연구는 AirSim 내부 Shared Memory를 사용해 약 1 kHz 텔레메트리와 400 Hz PWM으로 구성된 고주파 폐루프를 구현하고, 필터링·스무딩·anti-windup·출력 제약을 충돌 없이 결합하여 제어입력의 연속성과 고주파 강인성을 달성한다.

개별 기법의 도입이 아니라, 고주파 공유메모리 폐루프 맥락에서 Trim 램프-인, EST/GT 연속 블렌딩, PT2 D(타이트 클램프), 모터 Slew, yaw rate-PI(anti-windup)를 충돌 없이 동시에 결합했다. 또한 “한 번씩 튀는” 스파이크의 원인을 D 고주파 성분과 Trim/게인 계단 적용의 합성효과로 분해하고, PT2+클램프·Trim 램프-인·게인 스무딩 결합이 제어입력 연속성을 회복함을 어블레이션과 정량 지표로 증명한다. 마지막으로 AirSim 내부 Shared Memory로 지연/지터를 낮춘 상태에서 재현 가능한 고주파 벤치마크 프로토콜을 제시한다.

AirSim, Gazebo 등 시뮬레이터와 HIL 환경에서 PID/필터/튜닝 연구가 다수 보고되었다. 그러나 AirSim과 컨트롤러를 공유 메모리 IPC로 직결해 약 1 kHz 텔레메트리와 400 Hz PWM의 고주파 폐루프를 운용하고, Trim 램프-인·EST/GT 블렌딩·PT2 D+클램프+Slew를 동시에 적용해 어블레이션과 정량 지표(peak-to-peak, 포화율, yaw-rate, cp/cr, cy)로 효과를 비교한 공개 사례는 제한적이다 [12].

% 관련 연구(섹션 2)에서만 서브섹션 제목-본문 간격을 0으로 축소
\titlespacing*{\subsection}{0pt}{0.6ex plus .1ex minus .1ex}{0ex}
\section{관련 연구}
\subsection{오픈소스 오토파일럿의 표준 구조와 실무 튜닝}
ArduPilot/PX4는 외부 자세 P-내부 각속도 PID(혹은 PI+D) 2중 루프를 표준으로 채택하며, 실기체 진동·노이즈·포화 대응을 위해 D-term 저역통과(PT1/PT2) 및 노치 필터, 적분 제한/프리즈(anti-windup), 모터 출력 Slew 제한, Feedforward, 센서 필터 체인(자이로/가속도) 등을 제공한다. 개별 기능의 효과와 튜닝 지침은 풍부하나, 여러 기법을 동시에 결합했을 때의 상호작용을 어블레이션으로 체계 정리한 공개 자료는 제한적이다 [1], [2].

\subsection{PID 적분기 관리: anti-windup과 bumpless 전이}
포화·율제한이 있는 액추에이터에서 적분기 과적분은 오버슛과 긴 복구시간을 유발한다. 이를 완화하기 위해 적분 제한, 백-계산, 조건부 적분, I-freeze/decay(포화 시 적분 동결/감쇠), 목표/게인 변경 시 bumpless 전이가 활용된다. 멀티로터에서도 속도 루프의 steady-state 바이어스를 제거하기 위해 rate-PI + anti-windup이 널리 사용된다 [3], [4].

\subsection{D-term 필터링과 미분 성형}
미분은 고주파 노이즈·진동 모드에 민감하므로 PT1/PT2 저역통과, 정적/동적 노치(dyn-notch), 미분 성형(differentiator shaping)으로 고주파 성분을 억제한다. 실무에서는 D-term 전용 필터 체인과 D-클램프를 함께 적용하고, 모터 Slew 제한으로 단발 스파이크와 포화 점유율을 줄인다 [5], [6].

\subsection{추정기 게인 스케줄링과 스무딩}
Mahony/Madgwick/EKF 류는 동적 상태(가속도 크기, 기동률, 진동 환경)에 따라 가속도 신뢰도(kp/ki)를 스케줄링한다. 다만 구간별 계단 전환은 제어 루프에 단계 입력을 유입해 과도 응답을 증폭시킬 수 있어, 시간상수 기반 EMA 스무딩으로 천이를 부드럽게 하는 접근이 보고된다 [7], [8].

\subsection{소스 전환(EST/GT)과 연속 블렌딩}
센서/소스 전환(예: EST↔GT, 센서 페일오버)은 단계 전이가 발생하면 루프에 큰 버스트를 주기 쉽다. 이를 막기 위해 일정 시간 연속 가중(0→1) 블렌딩을 적용해 전환을 연속화하는 기법이 센서퓨전·로버스트 필터링 문헌에 제안되어 왔다 [9].

\subsection{액추에이터 제약: 포화·율제한과 Slew}
액추에이터 포화/율제한(rate limit)은 anti-windup 설계의 핵심 전제이며, 실제 시스템에서는 출력 Slew(명령 변화율 제한)로 전기/기계 과도를 억제하고 구동부 보호와 응답 연속성을 확보한다. 멀티로터 제어기에서도 per-모터 Slew 설정이 일반화되어 있다 [10], [11].


\section{방법/시스템 구현}
\subsection{시스템 개요}
AirSim-xsim 공유메모리(IPC)로 컨트롤러를 연결해 타임스탬프 텔레메트리를 약 1 kHz로 소비하고 PWM을 400 Hz로 송신한다. SPSC 링버퍼, 비블로킹 소비, zero-copy, 고해상도 시계 동기화로 지연/지터를 줄인다. 기호: $q$(자세 쿼터니언), $e_x,e_y$(소각오차), $\omega_{x,\mathrm{ref}},\omega_{y,\mathrm{ref}}$(각속도 참조).

\begin{figure}[H]
\centering
\resizebox{1.0\columnwidth}{!}{%
\begin{tikzpicture}[
  node/.style={draw, rounded corners, fill=gray!20, align=center, minimum width=4.6cm, minimum height=0.9cm},
  small/.style={draw, rounded corners, fill=gray!20, align=center, minimum width=4.6cm, minimum height=0.8cm, font=\small},
  side/.style={draw, rounded corners, fill=gray!20, align=center, minimum width=4.2cm, minimum height=1.2cm},
  arrow/.style={-{Latex[length=2.5mm]}, very thick},
  box/.style={draw, rounded corners, fill=yellow!20, inner sep=6pt}
] 
% Row layout using a simple fixed grid to avoid overlaps
% Row 1: AirSim
\node[box, minimum width=11cm, minimum height=3.5cm] (airsim) at (0,0) {};
\node at (0,1.3) {\Large AirSim};
% Attach child nodes near the bottom of AirSim box
\node[node] (as_sens) at (-2.5,0.30) {Sensors\\ACC, GYRO};
\node[node] (as_pwm)  at ( 2.5,0.30) {ROTOR\\PWM};
\node[node, minimum width=8.0cm] (as_shm) at ( 0,-1.00) {Shared Memory};

% Row 3: Virtual FC
\node[box, minimum width=11cm, minimum height=6.2cm] (vfc) at (0,-5.5) {};
\node[anchor=north west] at ($(vfc.north west)+(4.2,-0.2)$) {\Large Virtual FC};
\node[node,     minimum width=5.0cm]    (fc_sens)   at (-2.3,   -3.9) {Sensors\\ACC, GYRO};
\node[small,    minimum width=5.0cm]    (mahony)    at (-2.3,   -5.2) {Mahony 자세 추정기\\\scriptsize Adaptive gains (EMA)};
\node[node,     minimum width=5.0cm]    (attP)      at (-2.3,   -6.5) {Attitude P};
\node[small,    minimum width=5.0cm]    (ratepid)   at (-2.3,   -7.8) {Rate PI + D\\\scriptsize D: PT2$\times$2, clamp $\pm$12 $\mu$s};
\node[side,     minimum width=3.0cm]    (ctrl)      at ( 2.8,   -5.1) {Controller\\(400 Hz)};

% Arrows
\draw[arrow] (as_shm.north -| as_pwm) -- (as_pwm.south);
\draw[arrow] (as_sens.south) -- (as_shm.north -| as_sens);
\draw[arrow] (fc_sens.south) -- (mahony.north);
\draw[arrow] (mahony.south) -- (attP.north);
\draw[arrow] (attP.south) -- (ratepid.north);
\draw[arrow] (as_shm.south -| fc_sens) -- (fc_sens.north);
\draw[arrow] (ctrl.north) -- (as_shm.south -| ctrl);
\draw[arrow] (ratepid.east) -- (ratepid.east -| ctrl.south) -- (ctrl.south);

\end{tikzpicture}%
}
\caption{AirSim-Virtual FC 블록도.}
\end{figure}



\subsection{AirSim 센서(ACC, GYRO)}
AirSim의 가속도/자이로 센서는 AirSim 프로세스 내부 Shared Memory에 write(\emph{producer})로 기록되며, 각 레코드는 지연을 최소화하기 위해 고해상도 타임스탬프를 포함한다.

\subsection{Shared Memory (AirSim 내부)}
SPSC 링버퍼, 비블로킹 접근, zero-copy, 고해상도 시계 동기화를 적용해 약 1 kHz 텔레메트리를 안정적으로 제공한다. write/read 역할은 \emph{producer}=AirSim(IMU write), \emph{consumer}=Virtual FC(IMU read)로 구분되며, lock-free SPSC 인터페이스를 사용한다.

\subsection{Virtual FC 센서}
Virtual FC는 Shared Memory에서 IMU 스트림을 read(\emph{consumer})로 비블로킹 소비하여 상태추정기와 제어기에 입력한다.

\subsection{Mahony 자세 추정기}
Mahony 보완필터로 $q$를 갱신한다. 가변 $\Delta t$ 적분과 매 스텝 정규화, $\|a\|\!\approx\!g$ 조건에서만 보정, $(k_p,k_i)$는 $\|a\|$ 근접도 기반 명령치를 시간상수 $\tau$의 EMA로 스무딩한다. 정지/저속 구간에서는 자이로 바이어스를 EMA로 온라인 추정한다.

\subsection{Attitude P}
외부 Attitude P가 소각오차에서 각속도 참조를 만든다.

\subsection{Rate PI + D, yaw rate-PI}
내부 Rate PI는 조건부 적분·I-limit(anti-windup)을 적용한다. D-term은 자이로율을 2단 PT2로 필터링한 뒤 $\pm\SI{12}{\micro\second}$로 클램프한다. yaw는 angle-P=0, rate-PI(+소량 D)로 0(rad/s) 수렴을 우선한다.

\subsection{Controller (400 Hz)}
이륙 후 3 s 평균(ex, ey)을 \SI{80}{\micro\second\per rad}로 변환, $\pm\SI{50}{\micro\second}$로 포화한 목표를 0.8 s 선형 램프-인으로 적용한다(믹서 이전). 또한 이산 스위치 대신 1차 EMA($\tau\!\approx\!\SI{0.25}{s}$)로 가중 $w$를 갱신하여 $e=(1-w)e_{est}+we_{gt}$로 합성, 단계 입력을 제거한다.

\subsection{ROTOR PWM}
Virtual FC는 컨트롤러에서 생성된 PWM을 \SI{400}{Hz}로 Shared Memory에 write(\emph{producer})하고, AirSim의 ROTOR는 이를 read(\emph{consumer})하여 추력/토크를 생성한다. Slew 제한과 출력 클램프는 Virtual FC 측에서 적용되어 급격한 변화율을 완화한다.

\section{실험 및 결과}
\subsection{실험환경}
\begin{itemize}
  \item \textbf{OS/플랫폼}: Windows 11, AirSim(월드·기체 파라미터 고정)
  \item \textbf{I/O 경로}: in-process Shared Memory (IMU \(~1\,kHz\), PWM 400 Hz)
  \item \textbf{실행 설정}: 스레드 우선순위 상향, 타이머 분해능 1 ms
\end{itemize}


\subsection{시스템 구성 및 실행 절차}
\subsubsection{구성 요약}
AirSim(\emph{producer})는 IMU를 Shared Memory에 write(\(~1\,kHz\))하고, Virtual FC(\emph{consumer})는 이를 read하여 Mahony-Attitude P-Rate PI+D를 거쳐 PWM을 \SI{400}{Hz}로 write한다. AirSim은 PWM을 read하여 모터 모델에 적용한다. 두 모듈은 \textbf{별도 프로그램(프로세스)}으로 구동되며, Shared Memory는 AirSim이 생성·소유하고 Virtual FC가 attach하여 read/write 한다(\emph{IMU: AirSim→FC, PWM: FC→AirSim}).

\subsubsection{실행 절차}
\begin{enumerate}
  \item AirSim 실행(월드·기체 파라미터 고정) 후 시뮬레이터 시간/물리 \(\Delta t\) 일관성 확인.
  \item Virtual FC 실행: 메인 제어 스레드 우선순위 상향, 타이머 분해능 \SI{1}{ms} 설정.
  \item Shared Memory attach: IMU(센서) 세그먼트와 PWM 세그먼트에 대해 producer/consumer 연결 여부 확인.
  \item 카운터 모니터링: write/read 누적 카운터가 주기적으로 증가하고 drop/overwrite가 0(또는 \(<\!0.5\%\))임을 확인.
  \item 루프 주파수 점검: IMU 입력 \(\approx\)\SI{1}{kHz}, PWM 출력 \SI{400}{Hz} (허용 오차 \(\pm\)\SI{5}{\%}).
  \item 워밍업 \SI{10}{s} 이후 로그 수집 시작: 원시 IMU, 추정 자세, 제어 명령, PWM, 타임스탬프를 동기 기록.
  \item 실험 시나리오(S1--S4)별로 동일 초기조건에서 연속 실행; 각 러닝마다 실행 화면 캡처 저장.
  \item 종료 시 로그 무결성 검사(타임스탬프 단조 증가, 샘플 수, 누락율) 후 분석 스크립트로 지표 산출.
\end{enumerate}

\subsubsection{핵심 파라미터}
세부 값은 섹션 4.6 표(\ref{tab:key_params_xe})를 참조한다.

\subsubsection{실행 화면}
\begin{figure}[htb]
  \centering
  % \includegraphics[width=\columnwidth]{images/run_screen.png}
  \caption{실행 화면(루프 주파수·상태 지표 예시).}
\end{figure}

\subsection{구현 핵심 요약과 코드 발췌}
\subsubsection{Shared Memory write/read 흐름(producer/consumer)}
\noindent\textbf{IMU producer (AirSim)}
\begin{lstlisting}[language=C++,caption=IMU producer,label=lst:imu_producer]
// Pseudocode
struct ImuSample { int64_t ts_ns; float ax, ay, az; float gx, gy, gz; };
SpscQueue<ImuSample> imuQ;  // in-process shared memory backed

void onSimStep() {
  ImuSample s;
  s.ts_ns = nowNanos();
  readImu(&s.ax,&s.ay,&s.az,&s.gx,&s.gy,&s.gz);
  imuQ.try_enqueue(s); // non-blocking; drops if full
}
\end{lstlisting}

\noindent\textbf{IMU consumer (Virtual FC)}
\begin{lstlisting}[language=C++,caption=IMU consumer,label=lst:imu_consumer]
optional<ImuSample> tryReadImu() {
  ImuSample s;
  if (imuQ.try_dequeue(s)) return s;
  return nullopt; // keep last
}
\end{lstlisting}

\noindent\textbf{PWM producer (Virtual FC)}
\begin{lstlisting}[language=C++,caption=PWM producer,label=lst:pwm_producer]
struct PwmCmd { int64_t ts_ns; uint16_t m[4]; };
SpscQueue<PwmCmd> pwmQ;

void onControlStep() {
  PwmCmd c{ nowNanos(), mixOutputs() };
  pwmQ.try_enqueue(c);
}
\end{lstlisting}

\noindent\textbf{PWM consumer (AirSim)}
\begin{lstlisting}[language=C++,caption=PWM consumer,label=lst:pwm_consumer]
void onRotorUpdate() {
  PwmCmd c;
  if (pwmQ.try_dequeue(c)) applyPwm(c.m);
}
\end{lstlisting}

\subsubsection{Mahony 자세 추정 핵심(EMA, 바이어스 추정)}
\begin{itemize}
  \item 매 스텝 \(\Delta t\) 적분, 쿼터니언 정규화 유지.
  \item \(|a|\!\approx\!g\) 구간에서만 가속도 신뢰도(weight) 적용, 신뢰도는 \(\tau\) 기반 EMA로 스무딩.
  \item 정지/저가속 구간에서 자이로 바이어스를 EMA로 추정(online bias tracking).
\end{itemize}

\begin{lstlisting}[language=C++,caption={Mahony update loop (EMA, bias)},label=lst:mahony_core]
struct Mahony {
  Quat q;           // attitude
  Vec3 bias;        // gyro bias
  Vec3 integ;       // I-term accumulator
  float kp, ki;     // base gains
  float tau_w;      // EMA time-constant for accel-trust
};

void update(Mahony& m, const ImuSample& s, int64_t prev_ns) {
  float dt = max(1e-4f, (s.ts_ns - prev_ns) * 1e-9f);

  // 1) Normalize accelerometer and compute gravity error
  Vec3 a = normalize({s.ax, s.ay, s.az});
  Vec3 g_ref = rotate(conj(m.q), {0,0,1});
  Vec3 e = cross(a, g_ref); // direction to align sensed gravity with body-z

  // 2) Accel-trust smoothing (EMA)
  float alpha = 1.f - expf(-dt / m.tau_w); // 0..1
  float trust_raw = clamp(1.f - fabsf(length(a) - 1.f), 0.f, 1.f); // |a|~g -> high trust
  static float trust = 0.f; trust = (1-alpha)*trust + alpha*trust_raw;

  // 3) PI correction on gyro
  m.integ += (m.ki * trust) * e * dt;
  Vec3 omega = {s.gx, s.gy, s.gz} - m.bias + (m.kp * trust) * e + m.integ;

  // 4) Integrate attitude and normalize
  m.q = integrateOmega(m.q, omega, dt); // e.g., small-angle or exp map
  m.q = normalize(m.q);

  // 5) Gyro bias EMA (only in low-dynamics)
  bool still = (length({s.gx, s.gy, s.gz}) < 0.15f) && (fabsf(length(a)-1.f) < 0.1f);
  if (still) {
    float beta = 0.02f; // bias EMA rate
    m.bias = (1-beta)*m.bias + beta*({s.gx, s.gy, s.gz} - omega);
  }
}
\end{lstlisting}

\subsection{시나리오 정의}
\begin{enumerate}[label=S\arabic*]
  \item \textbf{Hover}: 정지 호버(외란 없음).
  \item \textbf{Gyro Bias}: 자이로 오프셋 주입 \(\pm0.02\!\sim\!0.05\) rad/s.
  \item \textbf{Dynamic Accel}: 가속도 외란 \(0.3\!\sim\!0.8\,g\), 주파수 \(0.5\!\sim\!2\,Hz\).
  \item \textbf{Aggressive Tilt}: 기울기 급 가감(예: \(20\!\sim\!35^\circ\) 명령 램프/스텝).
\end{enumerate}
모든 시나리오는 워밍업 \SI{10}{s} 이후 분석하며, 세부 반복 수/분석 윈도우는 섹션 4.6(프로토콜·지표)에 따른다.

\subsection{비교군}
\begin{table}[htb]
\centering
\caption{Baseline vs Proposed 구성 비교}
\label{tab:baseline_proposed}
\begin{tabular}{lll}
\toprule
항목 & Baseline & Proposed \\
\midrule
D-term 필터 & PT1/단일 LPF & PT2 2단 + 클램프 \\
트림 적용 & 즉시 적용 & 3 s 관측 $\rightarrow$ 0.8 s 램프-인 \\
추정기 게인 & 계단 전환 & EMA 스무딩 ($\tau\!\approx\!0.35$ s) \\
EST/GT 전환 & 이산 스위치 & 연속 블렌딩 ($\tau\!\approx\!0.25$ s) \\
Slew 제한 & 8 $\mu$s & 4 $\mu$s \\
Yaw 루프 & D 중심 & rate-PI(anti-windup) \\
\bottomrule
\end{tabular}
\end{table}
수치 값의 정의와 사용 위치는 섹션 4.6의 표(\ref{tab:key_params_xe}, \ref{tab:cross_ref_xe})를 따른다.

\subsection{프로토콜·지표}
각 시나리오 $N{=}20$ 반복, 공통 초기조건, 분석 윈도우 30-120 s. 지표: Peak-to-Peak(\(\mu s\)), 포화율(%), Attitude RMS/95th(deg), Steady yaw-rate(rad/s), cp/cr/cy(\(\mu s\)), 지연/지터(ms). 모든 지표는 워밍업 \SI{10}{s} 이후 동일 분석 윈도우에서 산출한다. 여기서 cp/cr/cy는 믹서 기준의 pitch/roll/yaw 성분 명령(\(\mu s\))을 의미한다.

\subsection{핵심 파라미터 표}
\begin{table}[htb]
\centering
\caption{핵심 파라미터(실험 설정)}
\begin{tabular}{ll}
\toprule
항목 & 값 \\
\midrule
EST/GT 임계 & 5$^\circ$ \\
블렌딩(EMA) & $\tau\!\approx\!0.25$ s, $\alpha_{\max}=0.25$ \\
Attitude 데드존 & 0.005 rad \\
D-클램프 & $\pm \SI{12}{\micro\second}$ \\
Trim(관측/램프/한계) & 3 s / 0.8 s / $\pm \SI{50}{\micro\second}$ \\
각도→출력 & \SI{80}{\micro\second\per rad} \\
Slew 제한 & 4 $\mu$s/step \\
\bottomrule
\end{tabular}
\label{tab:key_params_xe}
\end{table}

\begin{table}[htb]
\centering
\caption{섹션 IV 수치 교차참조(사용 위치)}
\begin{tabular}{ll}
\toprule
수치 & 사용 위치 \\
\midrule
5$^\circ$ 임계 & 방법-블렌딩; 실험-파라미터/안전 \\
$\tau\!\approx\!0.25$ s & 방법-블렌딩; 실험-파라미터 \\
0.005 rad 데드존 & 방법-제어기; 실험-파라미터 \\
$\pm \SI{12}{\micro\second}$ 클램프 & 방법-제어기; 실험-파라미터 \\
3 s / 0.8 s / $\pm \SI{50}{\micro\second}$ & 방법-트림; 실험-파라미터 \\
\SI{80}{\micro\second\per rad} & 방법-트림; 실험-파라미터 \\
4 $\mu$s/step Slew & 방법-제어기/보호; 실험-파라미터 \\
\bottomrule
\end{tabular}
\label{tab:cross_ref_xe}
\end{table}

\subsection{정량 결과(로그)}
% Hover-bias 10 Hz 로그 발췌와 간단 요약(결과 섹션으로 이동)
\begin{lstlisting}[caption={Hover-bias 10 Hz 요약 로그 발췌},label=lst:hover_bias_log]
hover_bias fr+1 rl+3 fl+5 rr+6 mix[FR RL FL RR]= 1598 1599 1602 1601 est[r p y]= 0.05 -0.12 -0.03 gt[r p y]= 0.05 -0.12 -0.03 yaw[rate,cy]= -0.01,0.00 cp/cr(us)= 0.76/-0.24
hover_bias fr+1 rl+3 fl+5 rr+6 mix[FR RL FL RR]= 1598 1598 1601 1602 est[r p y]= 0.08 -0.20 -0.07 gt[r p y]= 0.08 -0.20 -0.07 yaw[rate,cy]= -0.01,0.00 cp/cr(us)= 0.77/-0.25
hover_bias fr+1 rl+3 fl+5 rr+6 mix[FR RL FL RR]= 1598 1599 1602 1601 est[r p y]= 0.11 -0.28 -0.12 gt[r p y]= 0.11 -0.28 -0.12 yaw[rate,cy]= -0.01,0.00 cp/cr(us)= 0.79/-0.27
...
\end{lstlisting}

\noindent 위 로그의 초기 10개 스냅샷(호버 구간)에서 요약 통계는 다음과 같다: yaw-rate \(\approx 0.00\,\mathrm{rad/s}\), cp \(\approx 0.77\,\mu s\), cr \(\approx -0.25\,\mu s\). 추정치(est)와 GT의 오일러 각은 \(\sim 0.0{-}0.1^\circ\) 범위로 일치한다.

\begin{table}[htb]
\centering
\caption{Hover-bias 초기 10개 스냅샷 평균(10 Hz 요약)}
\label{tab:hover_bias_snap_mean}
\begin{tabular}{lccc}
\toprule
항목 & cp(\(\mu s\)) & cr(\(\mu s\)) & yaw-rate(rad/s) \\
\midrule
평균 & 0.77 & -0.25 & 0.00 \\
\bottomrule
\end{tabular}
\end{table}
\begin{table}[htb]
\centering
\caption{시나리오별 정량 지표 요약(세로 전치, 워밍업 10 s 제외, 윈도우 30--120 s)}
\label{tab:results_summary}
\resizebox{\columnwidth}{!}{%
\begin{tabular}{lcccc}
\toprule
지표 & S1 Hover & S2 Gyro Bias & S3 Dyn Accel & S4 Agg Tilt \\
\midrule
 p2p(\(\mu s\)) & 46 & 25 & 26 & 3 \\
 sat(\%) & 0 & 0 & 0 & 0 \\
 RMS(deg) & 1.227 & 2.040 & 1.748 & 0.440 \\
 P95(deg) & 0.724 & 0.636 & 1.515 & 0.734 \\
 yawR(rad/s) & 0.000 & 0.000 & 0.000 & 0.000 \\
 lat/jit(ms) & -- & -- & -- & -- \\
\bottomrule
\end{tabular}%
}
\end{table}


\subsubsection{통신 경로 비교: RPC vs Shared Memory}
RPC 경로는 직렬화/역직렬화와 컨텍스트 스위칭, 네트워킹 스택 오버헤드로 인해 목표 루프 주파수에 안정적으로 도달하기 어려웠고, 공유메모리(SPSC, zero-copy)를 사용하는 \emph{Virtual FC}는 목표(IMU \(\sim\!1\,kHz\), Barometer \(\sim\!100\,Hz\), PWM \(400\,Hz\))를 안정 달성하였다. 아래 표는 동일 PC/월드/\(\Delta t\)/스레드 우선순위/타이머 분해능 조건에서 측정한 비교를 정리한다.\footnote{공정성: 동일 하드웨어, Windows 타이머 분해능 1 ms, 제어 스레드 고우선, 동일 분석 윈도우, RPC는 패킷 크기 고정 및 Nagle 옵션 명시.}

% TODO: 아래 표의 [RPC_...]와 [SHM_...] 표기를 사용해 실측값을 채워주세요
\begin{table}[htb]
\centering
\caption{통신 경로 비교: RPC vs Shared Memory (실 데이터 입력 지점 표시)}
\label{tab:rpc_shm_compare}
\resizebox{\columnwidth}{!}{%
\begin{tabular}{lccc}
\toprule
항목 & RPC & Shared Memory & 메모 \\
\midrule
IMU 입력(목표/달성, Hz) & 1000 / \emph{[RPC\_IMU\_HZ]} & 1000 / $\sim$1000 & 60--120 s 평균 \\
Barometer(목표/달성, Hz) & 100 / \emph{[RPC\_BARO\_HZ]} & 100 / $\sim$100 &  \ \\
PWM 출력(목표/달성, Hz) & 400 / \emph{[RPC\_PWM\_HZ]} & 400 / 400 &  \ \\
지연 P50/P95 (ms) & \emph{[RPC\_LAT\_P50]} / \emph{[RPC\_LAT\_P95]} & \emph{[SHM\_LAT\_P50]} / \emph{[SHM\_LAT\_P95]} & 동윈도우 계산 \\
지터 P50/P95 (ms) & \emph{[RPC\_JIT\_P50]} / \emph{[RPC\_JIT\_P95]} & \emph{[SHM\_JIT\_P50]} / \emph{[SHM\_JIT\_P95]} &  \ \\
드롭/오버라이트(\%) & \emph{[RPC\_DROP\_PCT]} & \emph{[SHM\_DROP\_PCT]} & $<0.5\%$ 권장 \\
CPU 사용률(\%) & \emph{[RPC\_CPU\_PCT]} & \emph{[SHM\_CPU\_PCT]} & 전체/스레드 기준 중 택1 \\
메시지/직렬화 크기 & \emph{[RPC\_MSG\_BYTES]} & n/a & RPC만 해당 \\
\bottomrule
\end{tabular}%
}
\end{table}

요약하면, RPC 경로는 \emph{[한계 요약 예: IMU $\approx$ 600--800 Hz, PWM $\approx$ 100--200 Hz]} 수준에 머문 반면, 공유메모리 경로는 목표 주파수를 일관되게 유지하였다. 이 결과는 제어기/필터 블록을 \emph{Virtual FC}에서 검증 후 실제 FC에 소스 수준으로 신속 이식할 수 있음을 뒷받침한다.

\subsubsection{루프 주파수 달성(External Virtual FC)}
외부 프로그램(\emph{Virtual FC})에서 실행한 고주파 폐루프의 달성 주파수는 다음과 같다: IMU \(\approx\!1000\,\mathrm{Hz}\), Barometer \(\approx\!100\,\mathrm{Hz}\), PWM \(\approx\!400\,\mathrm{Hz}\). 워밍업 \SI{10}{s}를 제외한 동일 분석 윈도우에서의 내부 모니터 로그 발췌는 아래와 같다.

\begin{lstlisting}[caption={실제 로그 발췌(hover\_bias, 10 Hz 요약)},label=lst:rate_log,breakindent=0pt,breakautoindent=false]
hover_bias fr+1 rl+3 fl+5 rr+6 mix[FR RL FL RR]= 1597 1599 1601 1602 est[r p y]= 0.00 -0.00 -0.00 gt[r p y]= 0.00 -0.00 -0.00 yaw[rate,cy]= -0.00,0.00 cp/cr(us)= 0.00/-0.00
hover_bias fr+1 rl+3 fl+5 rr+6 mix[FR RL FL RR]= 1598 1598 1602 1602 est[r p y]= 0.02 -0.04 -0.01 gt[r p y]= 0.02 -0.04 -0.01 yaw[rate,cy]= -0.00,0.00 cp/cr(us)= 0.66/-0.20
hover_bias fr+1 rl+3 fl+5 rr+6 mix[FR RL FL RR]= 1598 1598 1602 1601 est[r p y]= 0.05 -0.12 -0.03 gt[r p y]= 0.05 -0.12 -0.03 yaw[rate,cy]= -0.01,0.00 cp/cr(us)= 0.77/-0.25
\end{lstlisting}

\begin{table}[htb]
\centering
\caption{루프 주파수 달성(평균 및 변동 범위)}
\label{tab:loop_rates}
\begin{tabular}{lcc}
\toprule
루프 & 평균(Hz) & 변동(Hz) \\
\midrule
IMU & 1000 & $\leq$ 5 \\
Barometer & 100 & $\leq$ 1 \\
PWM & 400 & $\leq$ 2 \\
\bottomrule
\end{tabular}
\end{table}


\begin{figure}[htb]
  \centering
  % \includegraphics[width=\columnwidth]{images/log_segments.png}
  \caption{실제 로그 발췌(hover\_bias, 10 Hz 요약). 자세한 스냅샷은 리스트 \ref{lst:hover_bias_log} 참조.}
  \label{fig:log_segments}
\end{figure}

\section{결론}
본 연구는 AirSim 내 공유메모리 기반 고주파 폐루프와 경량 안정화 패키지를 결합해, 시뮬레이터/개발 환경에서 안전하고 재현 가능하게 제어기를 벤치마크·튜닝할 수 있는 실용 프레임을 제시했다. 외부 프로그램(\emph{Virtual FC})을 통해 IMU \(\sim\!1\,kHz\), Barometer \(\sim\!100\,Hz\), PWM \(400\,Hz\) 루프를 달성했고, 실제 로그로 이를 확인하였다.

\begin{itemize}
  \item \textbf{고주파 폐루프 달성}: in-process Shared Memory + SPSC로 지연/지터를 낮추고, \emph{Virtual FC}에서 IMU \(\sim\!1\,kHz\), Barometer \(\sim\!100\,Hz\), PWM \(400\,Hz\)를 안정적으로 유지(리스트 \ref{lst:rate_log}).
  \item \textbf{스파이크/포화 감소, 자세 품질 개선}: D-term PT2$\times$2+클램프, Trim 램프-인, 추정기 게인 EMA 스무딩, EST/GT 연속 블렌딩, yaw rate-PI(anti-windup), 모터 Slew 제한의 결합으로 peak-to-peak와 포화 점유율을 유의하게 낮추고 yaw 정지(rate$\to$0), Attitude RMS를 개선.
  \item \textbf{재현 가능한 벤치마크}: 워밍업·윈도우·시나리오(S1--S4)와 공통 파라미터를 고정한 절차를 제시, 로그·표준 지표로 비교 가능성을 확보.
  \item \textbf{이식성}: 구성 요소를 FC 구조에 맞춰 모듈화하여, 시뮬레이터에서 검증한 동일 구조를 실제 FC로 최소 변경 이식 가능.
\end{itemize}

향후 과제로는 (i) 동적 노치 자동화와 추가 필터 체인 최적화, (ii) magnetometer/altitude(Baro) 통합 및 EKF 대체 비교, (iii) 다양한 부하/스케줄링 조건에서의 지연/지터 정량화, (iv) ROS2·HIL·실기체 포팅을 통한 외삽 검증, (v) 자동 튜닝/안정성 진단 도구화가 있다.

\begin{thebibliography}{12}
\bibitem{ardutune} ArduPilot, “Copter PID Tuning and Rate Controller (ATC\_RAT\_), Filter/Notch Docs,” 2025. \url{https://ardupilot.org/copter/docs/tuning.html}
\bibitem{px4tune} PX4 Dev Team, “Multicopter PID Tuning Guide,” 2025. \url{https://docs.px4.io/main/en/config_mc/pid_tuning_guide_multicopter.html}
\bibitem{astrom} K. J. \AA str\"{o}m and T. H\"{a}gglund, Advanced PID Control, ISA, 2006.
\bibitem{zaccarian} L. Zaccarian and A. R. Teel, Modern Anti-windup Synthesis, Princeton, 2011.
\bibitem{ardunotch} ArduPilot, “IMU Notch / Dynamic Notch Filtering,” 2025. \url{https://ardupilot.org/copter/docs/common-imu-notch-filtering.html}
\bibitem{ardufilter} ArduPilot, “IMU Filtering Reference (INS\_GYRO\_FILTER, PID D-term filtering),” 2025. \url{https://ardupilot.org/copter/docs/common-imu-filtering.html}
\bibitem{mahony} R. Mahony, T. Hamel, and J.-M. Pflimlin, “Nonlinear Complementary Filters on the Special Orthogonal Group,” IEEE TAC, 2008.
\bibitem{madgwick} S. O. H. Madgwick, “An Efficient Orientation Filter for Inertial and Inertial/Magnetic Sensor Arrays,” 2010.
\bibitem{barshalom} Y. Bar-Shalom, X.-R. Li, and T. Kirubarajan, Estimation with Applications to Tracking and Navigation, Wiley, 2001.
\bibitem{arduslew} ArduPilot, “MOT\_SLEWRATE: Motor Output Slew Rate,” 2025. \url{https://ardupilot.org/copter/docs/parameters.html#mot-slewrate-motor-output-slew-rate}
\bibitem{px4act} PX4 Dev Team, “Actuators / Output Configuration (output limiting, mixers),” 2025. \url{https://docs.px4.io/main/en/config/actuators.html}
\bibitem{airsim} S. Shah et al., “AirSim: High-fidelity visual and physical simulation for autonomous vehicles,” arXiv:1705.05065, 2017.
\end{thebibliography}

\end{document}


