\documentclass[a4paper,10pt,twocolumn]{article}

% XeLaTeX 한글 설정
\usepackage{fontspec}
\usepackage{kotex}
\setmainfont{Times New Roman}
\setsansfont{Arial}
\setmonofont{Courier New}
% 한글 글꼴 크기를 라틴과 맞추기 위해 스케일 조정
% 필요 시 0.88~1.00 사이로 미세 조정하세요
\setmainhangulfont[Scale=0.92]{Noto Serif CJK KR}
\setsanshangulfont[Scale=0.92]{Noto Sans CJK KR}

% 일반 패키지
\usepackage{graphicx}
\usepackage{booktabs}
\usepackage{siunitx}
\usepackage{hyperref}
\usepackage{tikz}
\usetikzlibrary{positioning,arrows.meta,fit,calc,matrix}
\usepackage{titling} % 제목 상단 여백 제어
\usepackage{titlesec} % 섹션 간격 제어
\usepackage{indentfirst} % 섹션/소섹션 직후 첫 문단도 들여쓰기
\usepackage{enumitem} % 리스트 간격 제어
\usepackage{caption} % 캡션 간격/서식 제어
\usepackage{listings} % 코드 블록(자동 줄바꿈)
\usepackage{float} % [H] 고정 배치를 위한 패키지
\usepackage{dblfloatfix} % 두 단(bottom) 플로트 허용
\usepackage{newunicodechar} % 유니코드 하이픈 대체
\usepackage{etoolbox} % 환경 진입 훅(표 전역 글자 크기 설정)
% 용지/여백 설정 (사용자 정의 210x280mm, 위20/아래15/좌우20, 머리말 간격 15, 제본 0)
\usepackage[
  paperwidth=210mm,
  paperheight=280mm,
  top=20mm,
  bottom=15mm,
  left=20mm,
  right=20mm,
  headsep=15mm,
  bindingoffset=0mm
]{geometry}
% 두 단 분리 시 각 단 너비 8 cm가 되도록 columnsep=10 mm (textwidth=170 mm 기준)
\setlength{\columnsep}{10mm}
% siunitx v3: replace deprecated detect-all
\sisetup{
  mode = match,
  propagate-math-font = true,
  reset-math-version = false,
  reset-text-family = false,
  reset-text-series = false,
  reset-text-shape = false,
  text-family-to-math = true,
  text-series-to-math = true
}

% 제목 상단 여백 줄이기 (필요 시 값 미세 조정)
\setlength{\droptitle}{-10mm}

% 섹션/소섹션 위아래 간격 축소 (before, after)
\titlespacing*{\section}{0pt}{0.8ex plus .2ex minus .1ex}{1.2ex}
\titlespacing*{\subsection}{0pt}{0.6ex plus .1ex minus .1ex}{0.4ex}

% 문단 간격/들여쓰기 조정 (촘촘한 본문)
\setlength{\parskip}{0.25ex plus 0.1ex minus 0.1ex}
\setlength{\parindent}{1.4em}

% 특수 하이픈(비분리 하이픈 U+2011 등)을 일반 하이픈으로 치환
% Map non-breaking hyphen(U+2011) and en dash(U+2013).
% Do NOT map ASCII '-' (causes newunicodechar error).
\newunicodechar{‑}{-} % U+2011 non-breaking hyphen
\newunicodechar{–}{-} % U+2013 (en dash -> ASCII hyphen)

% 제목 크기(폰트) 축소: 섹션/소섹션을 본문 크기로 맞춤
\titleformat{\section}{\bfseries\fontsize{12}{21.6}\selectfont\filcenter}{\thesection.}{0.5em}{}
\titleformat{\subsection}{\bfseries\normalsize}{\thesubsection.}{0.5em}{}
\titleformat{\subsubsection}{\bfseries\small}{\thesubsubsection.}{0.5em}{}

% 리스트(항목) 간격 축소
\setlist[itemize]{noitemsep, topsep=0.4ex, leftmargin=1.4em}
\setlist[enumerate]{noitemsep, topsep=0.4ex, leftmargin=1.4em}
% 그림 캡션과 본문(그림) 간격 축소
\captionsetup[figure]{skip=2pt, name=Fig., labelsep=period}
% 리스트 코드 캡션 좌측 정렬
\captionsetup[lstlisting]{singlelinecheck=false, justification=raggedright}
% 코드 기본 설정: 작은 폰트, 자동 줄바꿈, 고정폭, 여백 최소화
\lstset{
  basicstyle=\scriptsize\ttfamily,
  breaklines=true,
  breakatwhitespace=false,
  columns=fullflexible,
  keepspaces=true,
  frame=single,
  xleftmargin=0pt,
  xrightmargin=0pt
}

% 모든 표(table) 환경의 글자 크기를 일괄 더 축소(통일)
\AtBeginEnvironment{table}{\tiny\setlength{\tabcolsep}{1.5pt}\renewcommand{\arraystretch}{0.90}}

% 간단 키워드 매크로
\newcommand{\keywords}[1]{\par\noindent\textbf{키워드—} #1}
 % 제목 위 14pt 한줄(사용자 지정 문구). 비우면 출력되지 않음
\newcommand{\TitleTopLine}{}
 % 제목 아래 10pt 한줄(예: 저자명 등). 비우면 출력되지 않음
\newcommand{\TitleSubLine}{}
% 비어 있어도 고정 높이를 확보하는 박스
\newcommand{\TitleTopLineBox}{\noindent\parbox{\textwidth}{\centering{\fontsize{14}{22.4}\selectfont \rule{0pt}{22.4pt}\TitleTopLine}}}
\newcommand{\TitleSubLineBox}{\noindent\parbox{\textwidth}{\centering{\fontsize{10}{16}\selectfont \rule{0pt}{16pt}\TitleSubLine}}}
% 영문 제목 아래 10pt, 160% 줄간(=16pt) 빈 줄
\newcommand{\EngTitleSpacer}{\noindent\parbox{\textwidth}{\centering{\fontsize{10}{16}\selectfont \rule{0pt}{16pt}}}}
% 저자명: 11pt, line-height 190% (=20.9pt), 세로 중앙 정렬 (옵션 대괄호 불사용)
\newcommand{\AuthorLine}[1]{\noindent\vbox to 20.9pt{\vfil{\centering{\fontsize{11}{20.9}\selectfont #1}\par}\vfil}}
% 재사용 가능한 저자 정보 박스(어디서든 호출 가능)
\newcommand{\AuthorInfoBox}{%
  \noindent\fbox{\begin{minipage}{\columnwidth}\footnotesize
    Received: Mo \#\#, Year \quad Revised: Mo \#\#, Year \quad Accepted: Mo \#\#, Year\par
    \textsuperscript{1}Principal Research Engineer, \textsuperscript{2}Senior Research Engineer, \textsuperscript{3}Professor\par
    \textdagger\ Corresponding Author\par
    Tel: +82-\#\#-\#\#\#\#-\#\#\#\#, E-mail: ABCD@sase.or.kr\par
    ORCID: \#\#\#\#-\#\#\#\#-\#\#\#\#-\#\#\#\#\par
    Copyright \textcopyright\ The Society for Aerospace System Engineering
  \end{minipage}}%
}
% 현재 왼쪽 단 하단에 배치하는 헬퍼
\newcommand{\AuthorInfoBottomLeft}{\vspace*{\fill}\AuthorInfoBox}
% Figure reference helpers: sentence-begin vs in-text
\newcommand{\Figure}[1]{Figure~#1}
\newcommand{\Fig}[1]{Fig.~#1}

\title{AirSim 연동 고주파 제어 벤치마크·안정화 툴킷\\
AirSim-Integrated High-Frequency Control Benchmark · Stabilization Toolkit}

\author{김경목*\,\, 이남훈**\,\, 김성권**}
\date{}

\begin{document}
% JASE-style title/abstract block spanning both columns
\twocolumn[
\vspace*{-4mm}
\noindent\begin{tabular*}{\textwidth}{@{}l@{\extracolsep{\fill}}r@{}}
  {\fontsize{11}{13}\selectfont Research Paper} & {\fontsize{9}{11}\selectfont EISSN\,2508-7150} \\
  {\fontsize{10}{12}\selectfont\itshape Journal of Aerospace System Engineering} & {\fontsize{9}{11}\selectfont http://dx.doi.org/10.20910/JASE.\#\#\#\#.\#\#.\#\#} \\
  {\fontsize{10}{12}\selectfont Vol.\#, No.\#, pp.\#-\# (20\#\#)} & \\
\end{tabular*}
\rule{\textwidth}{0.4pt}
{\centering
\TitleTopLineBox\par\vspace{2pt}
{\fontsize{14.4}{14.4}\selectfont \bfseries AirSim 연동 고주파 제어 벤치마크·안정화 툴킷}\par
\TitleSubLineBox\par\vspace{2pt}
\AuthorLine{김경목\textsuperscript{1\textdagger}· 이남훈\textsuperscript{2}· 김성권\textsuperscript{3}}\par
{\fontsize{10}{12}\selectfont \textsuperscript{1}\, 서울과학기술대학교}\par
{\fontsize{10}{12}\selectfont \textsuperscript{2}\, 서울과학기술대학교}\par
{\fontsize{10}{12}\selectfont \textsuperscript{3}\, 서울과학기술대학교}\par\vspace{4pt}
{\large \bfseries AirSim-Integrated High-Frequency Control Benchmark · Stabilization Toolkit}\par
\EngTitleSpacer\par
\AuthorLine{Kyungmok Kim\textsuperscript{1\textdagger}· Namhoon Lee\textsuperscript{2}· Seongkwon Kim\textsuperscript{3}}\par
{\fontsize{10}{12}\selectfont \textsuperscript{1}\, Seoul National University of Science and Technology}\par
{\fontsize{10}{12}\selectfont \textsuperscript{2}\, Seoul National University of Science and Technology}\par
{\fontsize{10}{12}\selectfont \textsuperscript{3}\, Seoul National University of Science and Technology}\par
}
\vspace{5pt}
{\centering{\fontsize{10}{12}\selectfont\bfseries Abstract}\par}
\vspace{12pt}
{\fontsize{10}{12}\selectfont\ \ We present a high-rate simulation–control loop that connects AirSim to an external Software‑in‑the‑Loop (SIL) controller through an in-process shared-memory pipeline. AirSim publishes timestamped high‑rate sensor streams (kHz‑class inertial, 100‑Hz‑class environmental and GNSS) to a lock‑free ring buffer; the SIL controller consumes the streams, runs an attitude Proportional (P) and rate Proportional–Integral (PI) plus filtered Derivative (D) controller (PID structure), and returns PWM 400 Hz on the same path. Compared with Remote Procedure Call (RPC)‑based integration, the pipeline sustains target rates and markedly lowers latency and jitter. Under injected motor/gyro biases and dynamic accelerations, the loop reduces peak‑to‑peak spikes and saturation occupancy, achieves yaw stop (rate→0), and improves attitude root‑mean‑square (RMS). Because the SIL controller mirrors onboard flight controller (FC) structure, algorithms validated in AirSim transfer with minimal code changes. The result is a practical recipe and reference implementation for simulator and HIL development where high‑frequency sensing and actuation are essential.\par}\vspace{4pt}
{\centering{\fontsize{10}{12}\selectfont\bfseries 초\hspace{2em}록}\par}
\vspace{12pt}
{\fontsize{10}{12}\selectfont\ \ 본 논문은 AirSim과 Software‑in‑the‑Loop (SIL) 제어기를 프로세스 내 공유메모리로 직결한 고속 센싱·제어 폐루프를 제시한다. AirSim은 타임스탬프 기반의 고속 관성·환경·위치 센서를 링버퍼로 제공하고, SIL 제어기는 이를 소비해 자세 비례(Proportional, P)와 각속도 비례–적분(PI) 및 필터드 미분(Derivative, D) 제어(PID 구조)를 수행한 뒤 4채널 모터 PWM을 동일 공유메모리 채널을 통해 AirSim으로 전송한다. Remote Procedure Call (RPC) 대비 지연·지터가 감소하며 목표 주파수를 안정 유지한다. 모터/자이로 바이어스·동적 가속 환경에서 p2p·포화율이 줄고 yaw 정지(각속도→0)와 자세 root‑mean‑square (RMS)이 개선되며, 구조가 실기체 FC와 동일해 최소 변경 이식이 가능하다.\par}\vspace{12pt}
\noindent\textbf{Key Words:} AirSim, shared-memory pipeline, Software‑in‑the‑Loop (SIL) controller, high-rate sensing, attitude control, latency/jitter reduction
\vspace{4mm}
]

\section{서\hspace{2.0em}론}
\ \ 소형 멀티로터의 자세 제어는 모터·기체 비대칭, 센서 바이어스, 추정기 게인 스위칭 등으로 출력 스파이크와 장기적 yaw 드리프트가 빈번하다. 본 연구는 AirSim과 Software‑in‑the‑Loop (SIL) 제어기를 프로세스 내 공유메모리 파이프라인으로 직결하여, 타임스탬프 관성측정장치(Inertial Measurement Unit, IMU) 1000 Hz와 기압(Barometer) 100 Hz·자기(Magnetometer) 100 Hz·GNSS

\AuthorInfoBottomLeft

\noindent(위치) 10 Hz의 센서 스트림을 소비하고, ESC/로터에 전달되는 PWM(로터 구동 명령) 400 Hz를 생성하는 고속 센싱·제어 폐루프를 제안한다.
제안 루프는 SIL 제어기에 자세 P와 각속도(회전 속도) PI+D 제어기를 배치하고, yaw 각속도‑PI(anti‑windup), 자이로 D‑term PT2×2+클램프, 호버 트림 추정 후 램프‑인, Mahony 가속도 신뢰도 게인 스무딩, EST/GT 연속 블렌딩, 모터 Slew 제한을 결합한다.
RPC 기반 통합 대비 본 파이프라인은 지연·지터를 낮추면서 목표 주파수를 안정 유지하고, 모터/자이로 바이어스·동적 가속 조건에서 peak‑to‑peak와 포화율을 감소, yaw 정지(각속도→0)와 자세 root‑mean‑square (RMS) 지표를 개선한다. 또한 SIL 제어기가 실제 비행제어기(FC) 구조를 반영하므로 시뮬레이터에서 검증한 알고리즘을 최소 변경으로 실기체에 이식 가능하다.
AirSim, Gazebo 등에서 PID/필터/튜닝 연구는 다수 보고되었으나, AirSim↔공유메모리↔SIL 제어기로 고속 센싱·제어(IMU 1000 Hz, Baro/Mag 100 Hz, GNSS 10 Hz, PWM 400 Hz)를 일관 유지하며 RPC와 정량 비교(지연/지터, p2p, 포화율, yaw‑rate, RMS)를 제시한 공개 사례는 제한적이다 [12].
또한 제안한 공유메모리 기반 분리 구조는 SIL 기능을 비행제어기(FC) 개발과 모듈 경계로 분리하고 시뮬레이터에 독립적인 빌드·실행을 가능하게 하여, 컴파일 시간과 배포 복잡도를 줄이며 반복 개발 속도를 높인다.

% 관련 연구(섹션 2)에서만 서브섹션 제목-본문 간격을 0으로 축소
\titlespacing*{\subsection}{0pt}{0.6ex plus .1ex minus .1ex}{0ex}
\section{관련 연구}
\subsection{오픈소스 오토파일럿의 표준 구조와 실무 튜닝}
ArduPilot/PX4는 외부 자세 P-내부 각속도 PID(혹은 PI+D) 2중 루프를 표준으로 채택하며, 실기체 진동·노이즈·포화 대응을 위해 D-term 저역통과(PT1/PT2) 및 노치 필터, 적분 제한/프리즈(anti-windup), 모터 출력 Slew 제한, Feedforward, 센서 필터 체인(자이로/가속도) 등을 제공한다. 개별 기능의 효과와 튜닝 지침은 풍부하나, 여러 기법을 동시에 결합했을 때의 상호작용을 어블레이션으로 체계 정리한 공개 자료는 제한적이다 [1], [2].

\subsection{PID 적분기 관리: anti-windup과 bumpless 전이}
포화·율제한이 있는 액추에이터에서 적분기 과적분은 오버슛과 긴 복구시간을 유발한다. 이를 완화하기 위해 적분 제한, 백-계산, 조건부 적분, I-freeze/decay(포화 시 적분 동결/감쇠), 목표/게인 변경 시 bumpless 전이가 활용된다. 멀티로터에서도 속도 루프의 steady-state 바이어스를 제거하기 위해 rate-PI + anti-windup이 널리 사용된다 [3], [4].

\subsection{D-term 필터링과 미분 성형}
미분은 고주파 노이즈·진동 모드에 민감하므로 PT1/PT2 저역통과, 정적/동적 노치(dyn-notch), 미분 성형(differentiator shaping)으로 고주파 성분을 억제한다. 실무에서는 D-term 전용 필터 체인과 D-클램프를 함께 적용하고, 모터 Slew 제한으로 단발 스파이크와 포화 점유율을 줄인다 [5], [6].

\subsection{추정기 게인 스케줄링과 스무딩}
Mahony/Madgwick/EKF 류는 동적 상태(가속도 크기, 기동률, 진동 환경)에 따라 가속도 신뢰도(kp/ki)를 스케줄링한다. 다만 구간별 계단 전환은 제어 루프에 단계 입력을 유입해 과도 응답을 증폭시킬 수 있어, 시간상수 기반 EMA 스무딩으로 천이를 부드럽게 하는 접근이 보고된다 [7], [8].

\subsection{소스 전환(EST/GT)과 연속 블렌딩}
센서/소스 전환(예: EST↔GT, 센서 페일오버)은 단계 전이가 발생하면 루프에 큰 버스트를 주기 쉽다. 이를 막기 위해 일정 시간 연속 가중(0→1) 블렌딩을 적용해 전환을 연속화하는 기법이 센서퓨전·로버스트 필터링 문헌에 제안되어 왔다 [9].

\subsection{액추에이터 제약: 포화·출력 변화율 제한과 Slew}
액추에이터 포화/출력 변화율 제한(rate limit)은 anti-windup 설계의 핵심 전제이며, 실제 시스템에서는 출력 Slew(명령 변화율 제한)로 전기/기계 과도를 억제하고 구동부 보호와 응답 연속성을 확보한다. 멀티로터 제어기에서도 per-모터 Slew 설정이 일반화되어 있다 [10], [11].


\section{방법/시스템 구현}
\subsection{시스템 개요}
AirSim–공유메모리–SIL 제어기 구조로 연결해 타임스탬프 텔레메트리를 약 1 kHz로 소비하고 PWM을 400 Hz로 송신한다. SPSC 링버퍼, 비블로킹 소비, zero-copy, 고해상도 시계 동기화로 지연/지터를 줄인다. 기호: $q$(자세 쿼터니언), $e_x,e_y$(소각오차), $\omega_{x,\mathrm{ref}},\omega_{y,\mathrm{ref}}$(각속도 참조).

\begin{figure}[H]
\centering
\resizebox{1.0\columnwidth}{!}{%
\begin{tikzpicture}[
  node/.style={draw, rounded corners, fill=gray!20, align=center, minimum width=4.6cm, minimum height=0.9cm},
  small/.style={draw, rounded corners, fill=gray!20, align=center, minimum width=4.6cm, minimum height=0.8cm, font=\small},
  side/.style={draw, rounded corners, fill=gray!20, align=center, minimum width=4.2cm, minimum height=1.2cm},
  arrow/.style={-{Latex[length=2.5mm]}, very thick},
  box/.style={draw, rounded corners, fill=yellow!20, inner sep=6pt}
] 
% Row layout using a simple fixed grid to avoid overlaps
% Row 1: AirSim
\node[box, minimum width=11cm, minimum height=2.1cm] (airsim) at (0,0) {};
\node at (0,0.78) {\Large AirSim};
% Attach child nodes near the bottom of AirSim box
\node[node] (as_sens) at (-2.5,0.30) {Sensors\\ACC, GYRO};
\node[node] (as_pwm)  at ( 2.5,0.30) {ROTOR\\PWM};
% OS-level shared memory placed between AirSim and SIL
\node[node, minimum width=8.0cm] (shm) at ( 0,-2.40) {Shared Memory (OS)};

% Row 3: Virtual FC
\node[box, minimum width=11cm, minimum height=6.2cm] (vfc) at (0,-5.5) {};
\node[anchor=north west] at ($(vfc.north west)+(4.2,-0.2)$) {\Large SIL Controller};
\node[node,     minimum width=5.0cm]    (fc_sens)   at (-2.3,   -3.9) {Sensors\\ACC, GYRO};
\node[small,    minimum width=5.0cm]    (mahony)    at (-2.3,   -5.2) {Mahony 자세 추정기\\\scriptsize Adaptive gains (EMA)};
\node[node,     minimum width=5.0cm]    (attP)      at (-2.3,   -6.5) {Attitude P};
\node[small,    minimum width=5.0cm]    (ratepid)   at (-2.3,   -7.8) {Rate PI + D\\\scriptsize D: PT2$\times$2, clamp $\pm$12 $\mu$s};
\node[side,     minimum width=3.0cm]    (ctrl)      at ( 2.8,   -5.1) {Controller\\(400 Hz)};

% Arrows
\draw[arrow] (shm.north -| as_pwm) -- (as_pwm.south);
\draw[arrow] (as_sens.south) -- (shm.north -| as_sens);
\draw[arrow] (fc_sens.south) -- (mahony.north);
\draw[arrow] (mahony.south) -- (attP.north);
\draw[arrow] (attP.south) -- (ratepid.north);
\draw[arrow] (shm.south -| fc_sens) -- (fc_sens.north);
\draw[arrow] (ctrl.north) -- (shm.south -| ctrl);
\draw[arrow] (ratepid.east) -- (ratepid.east -| ctrl.south) -- (ctrl.south);

\end{tikzpicture}%
}
\caption{AirSim–Shared Memory–SIL block diagram.}\label{fig:airsim_sil}
\end{figure}



\subsection{AirSim 센서(ACC, GYRO)}
AirSim의 가속도/자이로 센서는 OS 공유메모리(IPC) 세그먼트에 write(\emph{producer})로 기록되며, 각 레코드는 지연을 최소화하기 위해 고해상도 타임스탬프를 포함한다.

\subsection{Shared Memory (OS IPC)}
SPSC 링버퍼, 비블로킹 접근, zero-copy, 고해상도 시계 동기화를 적용해 약 1 kHz 텔레메트리를 안정적으로 제공한다. write/read 역할은 \emph{producer}=AirSim(IMU write), \emph{consumer}=SIL 제어기(IMU read)로 구분되며, lock-free SPSC 인터페이스를 사용한다.

\subsection{SIL 제어기 센서}
SIL 제어기는 Shared Memory에서 IMU 스트림을 read(\emph{consumer})로 비블로킹 소비하여 상태추정기와 제어기에 입력한다.

\subsection{Mahony 자세 추정기}
Mahony 보완필터로 $q$를 갱신한다. 가변 $\Delta t$ 적분과 매 스텝 정규화, $\|a\|\!\approx\!g$ 조건에서만 보정, $(k_p,k_i)$는 $\|a\|$ 근접도 기반 명령치를 시간상수 $\tau$의 EMA로 스무딩한다. 정지/저속 구간에서는 자이로 바이어스를 EMA로 온라인 추정한다.

\subsection{Attitude P}
외부 Attitude P가 소각오차에서 각속도 참조를 만든다.

\subsection{Rate PI + D, yaw rate-PI}
내부 Rate PI는 조건부 적분과 I항 제한(I-limit, anti-windup)을 적용한다. 여기서 I항은 적분항(Integral, I-term)을 의미한다. D-term은 자이로율을 2단 PT2로 필터링한 뒤 $\pm\SI{12}{\micro\second}$로 클램프한다. yaw는 angle-P=0, rate-PI(+소량 D)로 0(rad/s) 수렴을 우선한다.

\subsection{Controller (400 Hz)}
이륙 후 3 s 평균(ex, ey)을 \SI{80}{\micro\second\per rad}로 변환, $\pm\SI{50}{\micro\second}$로 포화한 목표를 0.8 s 선형 램프-인으로 적용한다(믹서 이전). 또한 이산 스위치 대신 1차 EMA($\tau\!\approx\!\SI{0.25}{s}$)로 가중 $w$를 갱신하여 $e=(1-w)e_{est}+we_{gt}$로 합성, 단계 입력을 제거한다.

\subsection{ROTOR PWM}
SIL 제어기는 컨트롤러에서 생성된 PWM을 \SI{400}{Hz}로 Shared Memory에 write(\emph{producer})하고, AirSim의 ROTOR는 이를 read(\emph{consumer})하여 추력/토크를 생성한다. Slew 제한과 출력 클램프는 SIL 제어기 측에서 적용되어 급격한 변화율을 완화한다.

\section{실험 및 결과}
\subsection{실험환경}
\begin{itemize}
  \item \textbf{OS/플랫폼}: Windows 11, AirSim(월드·기체 파라미터 고정)
  \item \textbf{I/O 경로}: in-process Shared Memory (IMU \(~1\,kHz\), PWM 400 Hz)
  \item \textbf{실행 설정}: 스레드 우선순위 상향, 타이머 분해능 1 ms
\end{itemize}


\subsection{시스템 구성 및 실행 절차}
\subsubsection{구성 요약}
AirSim(\emph{producer})는 IMU를 Shared Memory에 write(\(~1\,kHz\))하고, SIL 제어기(\emph{consumer})는 이를 read하여 Mahony-Attitude P-Rate PI+D를 거쳐 PWM을 \SI{400}{Hz}로 write한다. AirSim은 PWM을 read하여 모터 모델에 적용한다. 두 모듈은 \textbf{별도 프로그램(프로세스)}으로 구동되며, Shared Memory는 AirSim이 생성·소유하고 SIL 제어기가 attach하여 read/write 한다(\emph{IMU: AirSim→SIL, PWM: SIL→AirSim}).

\subsubsection{실행 절차}
\begin{enumerate}
  \item AirSim 실행(월드·기체 파라미터 고정) 후 시뮬레이터 시간/물리 \(\Delta t\) 일관성 확인.
  \item SIL 제어기 실행: 메인 제어 스레드 우선순위 상향, 타이머 분해능 \SI{1}{ms} 설정.
  \item Shared Memory attach: IMU(센서) 세그먼트와 PWM 세그먼트에 대해 producer/consumer 연결 여부 확인.
  \item 카운터 모니터링: write/read 누적 카운터가 주기적으로 증가하고 drop/overwrite가 0(또는 \(<\!0.5\%\))임을 확인.
  \item 루프 주파수 점검: IMU 입력 \(\approx\)\SI{1}{kHz}, PWM 출력 \SI{400}{Hz} (허용 오차 \(\pm\)\SI{5}{\%}).
  \item 워밍업 \SI{10}{s} 이후 로그 수집 시작: 원시 IMU, 추정 자세, 제어 명령, PWM, 타임스탬프를 동기 기록.
  \item 실험 시나리오(S1--S4)별로 동일 초기조건에서 연속 실행; 각 러닝마다 실행 화면 캡처 저장.
  \item 종료 시 로그 무결성 검사(타임스탬프 단조 증가, 샘플 수, 누락율) 후 분석 스크립트로 지표 산출.
\end{enumerate}

\subsubsection{핵심 파라미터}
세부 값은 섹션 4.6 표(\ref{tab:key_params_xe})를 참조한다.

\subsubsection{실행 화면}
\begin{figure}[htb]
  \centering
  % \includegraphics[width=\columnwidth]{images/run_screen.png}
  \caption{실행 화면(루프 주파수·상태 지표 예시).}
\end{figure}

\subsection{구현 핵심 요약과 코드 발췌}
\subsubsection{Shared Memory write/read 흐름(producer/consumer)}
\noindent\textbf{IMU producer (AirSim)}
\begin{lstlisting}[language=C++,caption=IMU producer,label=lst:imu_producer]
// Pseudocode
struct ImuSample { int64_t ts_ns; float ax, ay, az; float gx, gy, gz; };
SpscQueue<ImuSample> imuQ;  // in-process shared memory backed

void onSimStep() {
  ImuSample s;
  s.ts_ns = nowNanos();
  readImu(&s.ax,&s.ay,&s.az,&s.gx,&s.gy,&s.gz);
  imuQ.try_enqueue(s); // non-blocking; drops if full
}
\end{lstlisting}

\noindent\textbf{IMU consumer (SIL controller)}
\begin{lstlisting}[language=C++,caption=IMU consumer,label=lst:imu_consumer]
optional<ImuSample> tryReadImu() {
  ImuSample s;
  if (imuQ.try_dequeue(s)) return s;
  return nullopt; // keep last
}
\end{lstlisting}

\noindent\textbf{PWM producer (SIL controller)}
\begin{lstlisting}[language=C++,caption=PWM producer,label=lst:pwm_producer]
struct PwmCmd { int64_t ts_ns; uint16_t m[4]; };
SpscQueue<PwmCmd> pwmQ;

void onControlStep() {
  PwmCmd c{ nowNanos(), mixOutputs() };
  pwmQ.try_enqueue(c);
}
\end{lstlisting}

\noindent\textbf{PWM consumer (AirSim)}
\begin{lstlisting}[language=C++,caption=PWM consumer,label=lst:pwm_consumer]
void onRotorUpdate() {
  PwmCmd c;
  if (pwmQ.try_dequeue(c)) applyPwm(c.m);
}
\end{lstlisting}

\subsubsection{Mahony 자세 추정 핵심(EMA, 바이어스 추정)}
\begin{itemize}
  \item 매 스텝 \(\Delta t\) 적분, 쿼터니언 정규화 유지.
  \item \(|a|\!\approx\!g\) 구간에서만 가속도 신뢰도(weight) 적용, 신뢰도는 \(\tau\) 기반 EMA로 스무딩.
  \item 정지/저가속 구간에서 자이로 바이어스를 EMA로 추정(online bias tracking).
\end{itemize}

\begin{lstlisting}[language=C++,caption={Mahony update loop (EMA, bias)},label=lst:mahony_core]
struct Mahony {
  Quat q;           // attitude
  Vec3 bias;        // gyro bias
  Vec3 integ;       // I-term accumulator
  float kp, ki;     // base gains
  float tau_w;      // EMA time-constant for accel-trust
};

void update(Mahony& m, const ImuSample& s, int64_t prev_ns) {
  float dt = max(1e-4f, (s.ts_ns - prev_ns) * 1e-9f);

  // 1) Normalize accelerometer and compute gravity error
  Vec3 a = normalize({s.ax, s.ay, s.az});
  Vec3 g_ref = rotate(conj(m.q), {0,0,1});
  Vec3 e = cross(a, g_ref); // direction to align sensed gravity with body-z

  // 2) Accel-trust smoothing (EMA)
  float alpha = 1.f - expf(-dt / m.tau_w); // 0..1
  float trust_raw = clamp(1.f - fabsf(length(a) - 1.f), 0.f, 1.f); // |a|~g -> high trust
  static float trust = 0.f; trust = (1-alpha)*trust + alpha*trust_raw;

  // 3) PI correction on gyro
  m.integ += (m.ki * trust) * e * dt;
  Vec3 omega = {s.gx, s.gy, s.gz} - m.bias + (m.kp * trust) * e + m.integ;

  // 4) Integrate attitude and normalize
  m.q = integrateOmega(m.q, omega, dt); // e.g., small-angle or exp map
  m.q = normalize(m.q);

  // 5) Gyro bias EMA (only in low-dynamics)
  bool still = (length({s.gx, s.gy, s.gz}) < 0.15f) && (fabsf(length(a)-1.f) < 0.1f);
  if (still) {
    float beta = 0.02f; // bias EMA rate
    m.bias = (1-beta)*m.bias + beta*({s.gx, s.gy, s.gz} - omega);
  }
}
\end{lstlisting}

\subsection{시나리오 정의}
\begin{enumerate}[label=S\arabic*]
  \item \textbf{Hover}: 정지 호버(외란 없음).
  \item \textbf{Gyro Bias}: 자이로 오프셋 주입 \(\pm0.02\!\sim\!0.05\) rad/s.
  \item \textbf{Dynamic Accel}: 가속도 외란 \(0.3\!\sim\!0.8\,g\), 주파수 \(0.5\!\sim\!2\,Hz\).
  \item \textbf{Aggressive Tilt}: 기울기 급 가감(예: \(20\!\sim\!35^\circ\) 명령 램프/스텝).
\end{enumerate}
모든 시나리오는 워밍업 \SI{10}{s} 이후 분석하며, 세부 반복 수/분석 윈도우는 섹션 4.6(프로토콜·지표)에 따른다.

\subsection{비교군}
\begin{table}[htb]
\centering
\caption{Baseline vs Proposed 구성 비교}
\label{tab:baseline_proposed}
\begin{tabular}{lll}
\toprule
항목 & Baseline & Proposed \\
\midrule
D-term 필터 & PT1/단일 LPF & PT2 2단 + 클램프 \\
트림 적용 & 즉시 적용 & 3 s 관측 $\rightarrow$ 0.8 s 램프-인 \\
추정기 게인 & 계단 전환 & EMA 스무딩 ($\tau\!\approx\!0.35$ s) \\
EST/GT 전환 & 이산 스위치 & 연속 블렌딩 ($\tau\!\approx\!0.25$ s) \\
Slew 제한 & 8 $\mu$s & 4 $\mu$s \\
Yaw 루프 & D 중심 & rate-PI(anti-windup) \\
\bottomrule
\end{tabular}
\end{table}
수치 값의 정의와 사용 위치는 섹션 4.6의 표(\ref{tab:key_params_xe}, \ref{tab:cross_ref_xe})를 따른다.

\subsection{프로토콜·지표}
각 시나리오 $N{=}20$ 반복, 공통 초기조건, 분석 윈도우 30-120 s. 지표: Peak-to-Peak(\(\mu s\)), 포화율(%), Attitude RMS/95th(deg), Steady yaw-rate(rad/s), cp/cr/cy(\(\mu s\)), 지연/지터(ms). 모든 지표는 워밍업 \SI{10}{s} 이후 동일 분석 윈도우에서 산출한다. 여기서 cp/cr/cy는 믹서 기준의 pitch/roll/yaw 성분 명령(\(\mu s\))을 의미한다.

\subsection{핵심 파라미터 표}
\begin{table}[htb]
\centering
\caption{핵심 파라미터(실험 설정)}
\begin{tabular}{ll}
\toprule
항목 & 값 \\
\midrule
EST/GT 임계 & 5$^\circ$ \\
블렌딩(EMA) & $\tau\!\approx\!0.25$ s, $\alpha_{\max}=0.25$ \\
Attitude 데드존 & 0.005 rad \\
D-클램프 & $\pm \SI{12}{\micro\second}$ \\
Trim(관측/램프/한계) & 3 s / 0.8 s / $\pm \SI{50}{\micro\second}$ \\
각도→출력 & \SI{80}{\micro\second\per rad} \\
Slew 제한 & 4 $\mu$s/step \\
\bottomrule
\end{tabular}
\label{tab:key_params_xe}
\end{table}

\begin{table}[htb]
\centering
\caption{섹션 IV 수치 교차참조(사용 위치)}
\begin{tabular}{ll}
\toprule
수치 & 사용 위치 \\
\midrule
5$^\circ$ 임계 & 방법-블렌딩; 실험-파라미터/안전 \\
$\tau\!\approx\!0.25$ s & 방법-블렌딩; 실험-파라미터 \\
0.005 rad 데드존 & 방법-제어기; 실험-파라미터 \\
$\pm \SI{12}{\micro\second}$ 클램프 & 방법-제어기; 실험-파라미터 \\
3 s / 0.8 s / $\pm \SI{50}{\micro\second}$ & 방법-트림; 실험-파라미터 \\
\SI{80}{\micro\second\per rad} & 방법-트림; 실험-파라미터 \\
4 $\mu$s/step Slew & 방법-제어기/보호; 실험-파라미터 \\
\bottomrule
\end{tabular}
\label{tab:cross_ref_xe}
\end{table}

\subsection{정량 결과(로그)}
% Hover-bias 10 Hz 로그 발췌와 간단 요약(결과 섹션으로 이동)
\begin{lstlisting}[caption={Hover-bias 10 Hz 요약 로그 발췌},label=lst:hover_bias_log]
hover_bias fr+1 rl+3 fl+5 rr+6 mix[FR RL FL RR]= 1598 1599 1602 1601 est[r p y]= 0.05 -0.12 -0.03 gt[r p y]= 0.05 -0.12 -0.03 yaw[rate,cy]= -0.01,0.00 cp/cr(us)= 0.76/-0.24
hover_bias fr+1 rl+3 fl+5 rr+6 mix[FR RL FL RR]= 1598 1598 1601 1602 est[r p y]= 0.08 -0.20 -0.07 gt[r p y]= 0.08 -0.20 -0.07 yaw[rate,cy]= -0.01,0.00 cp/cr(us)= 0.77/-0.25
hover_bias fr+1 rl+3 fl+5 rr+6 mix[FR RL FL RR]= 1598 1599 1602 1601 est[r p y]= 0.11 -0.28 -0.12 gt[r p y]= 0.11 -0.28 -0.12 yaw[rate,cy]= -0.01,0.00 cp/cr(us)= 0.79/-0.27
...
\end{lstlisting}

\noindent 위 로그의 초기 10개 스냅샷(호버 구간)에서 요약 통계는 다음과 같다: yaw-rate \(\approx 0.00\,\mathrm{rad/s}\), cp \(\approx 0.77\,\mu s\), cr \(\approx -0.25\,\mu s\). 추정치(est)와 GT의 오일러 각은 \(\sim 0.0{-}0.1^\circ\) 범위로 일치한다.

\begin{table}[htb]
\centering
\caption{Hover-bias 초기 10개 스냅샷 평균(10 Hz 요약)}
\label{tab:hover_bias_snap_mean}
\begin{tabular}{lccc}
\toprule
항목 & cp(\(\mu s\)) & cr(\(\mu s\)) & yaw-rate(rad/s) \\
\midrule
평균 & 0.77 & -0.25 & 0.00 \\
\bottomrule
\end{tabular}
\end{table}
\begin{table}[htb]
\centering
\caption{시나리오별 정량 지표 요약(세로 전치, 워밍업 10 s 제외, 윈도우 30--120 s)}
\label{tab:results_summary}
\resizebox{\columnwidth}{!}{%
\begin{tabular}{lcccc}
\toprule
지표 & S1 Hover & S2 Gyro Bias & S3 Dyn Accel & S4 Agg Tilt \\
\midrule
 p2p(\(\mu s\)) & 46 & 25 & 26 & 3 \\
 sat(\%) & 0 & 0 & 0 & 0 \\
 RMS(deg) & 1.227 & 2.040 & 1.748 & 0.440 \\
 P95(deg) & 0.724 & 0.636 & 1.515 & 0.734 \\
 yawR(rad/s) & 0.000 & 0.000 & 0.000 & 0.000 \\
 lat/jit(ms) & -- & -- & -- & -- \\
\bottomrule
\end{tabular}%
}
\end{table}


\subsubsection{통신 경로 비교: RPC vs Shared Memory}
RPC 경로는 직렬화/역직렬화와 컨텍스트 스위칭, 네트워킹 스택 오버헤드로 인해 목표 루프 주파수에 안정적으로 도달하기 어려웠고, 공유메모리(SPSC, zero-copy)를 사용하는 \emph{SIL 제어기}는 목표(IMU \(\sim\!1\,kHz\), Barometer \(\sim\!100\,Hz\), PWM \(400\,Hz\))를 안정 달성하였다. 아래 표는 동일 PC/월드/\(\Delta t\)/스레드 우선순위/타이머 분해능 조건에서 측정한 비교를 정리한다.\footnote{공정성: 동일 하드웨어, Windows 타이머 분해능 1 ms, 제어 스레드 고우선, 동일 분석 윈도우, RPC는 패킷 크기 고정 및 Nagle 옵션 명시.}

% TODO: 아래 표의 [RPC_...]와 [SHM_...] 표기를 사용해 실측값을 채워주세요
\begin{table}[htb]
\centering
\caption{통신 경로 비교: RPC vs Shared Memory (실 데이터 입력 지점 표시)}
\label{tab:rpc_shm_compare}
\resizebox{\columnwidth}{!}{%
\begin{tabular}{lccc}
\toprule
항목 & RPC & Shared Memory & 메모 \\
\midrule
Inertial Measurement Unit (IMU) 입력(목표/달성, Hz) & 1000 / \emph{[RPC\_IMU\_HZ]} & 1000 / $\sim$1000 & 60--120 s 평균 \\
Barometer(목표/달성, Hz) & 100 / \emph{[RPC\_BARO\_HZ]} & 100 / $\sim$100 &  \ \\
PWM 출력(목표/달성, Hz) & 400 / \emph{[RPC\_PWM\_HZ]} & 400 / 400 &  \ \\
지연 P50/P95 (ms) & \emph{[RPC\_LAT\_P50]} / \emph{[RPC\_LAT\_P95]} & \emph{[SHM\_LAT\_P50]} / \emph{[SHM\_LAT\_P95]} & 동윈도우 계산 \\
지터 P50/P95 (ms) & \emph{[RPC\_JIT\_P50]} / \emph{[RPC\_JIT\_P95]} & \emph{[SHM\_JIT\_P50]} / \emph{[SHM\_JIT\_P95]} &  \ \\
드롭/오버라이트(\%) & \emph{[RPC\_DROP\_PCT]} & \emph{[SHM\_DROP\_PCT]} & $<0.5\%$ 권장 \\
CPU 사용률(\%) & \emph{[RPC\_CPU\_PCT]} & \emph{[SHM\_CPU\_PCT]} & 전체/스레드 기준 중 택1 \\
메시지/직렬화 크기 & \emph{[RPC\_MSG\_BYTES]} & n/a & RPC만 해당 \\
\bottomrule
\end{tabular}%
}
\end{table}

요약하면, RPC 경로는 \emph{[한계 요약 예: IMU $\approx$ 600--800 Hz, PWM $\approx$ 100--200 Hz]} 수준에 머문 반면, 공유메모리 경로는 목표 주파수를 일관되게 유지하였다. 이 결과는 제어기/필터 블록을 \emph{SIL 제어기}에서 검증 후 실제 FC에 소스 수준으로 신속 이식할 수 있음을 뒷받침한다.

\subsubsection{루프 주파수 달성(External Virtual FC)}
외부 프로그램(\emph{Virtual FC})에서 실행한 고주파 폐루프의 달성 주파수는 다음과 같다: IMU \(\approx\!1000\,\mathrm{Hz}\), Barometer \(\approx\!100\,\mathrm{Hz}\), PWM \(\approx\!400\,\mathrm{Hz}\). 워밍업 \SI{10}{s}를 제외한 동일 분석 윈도우에서의 내부 모니터 로그 발췌는 아래와 같다.

\begin{lstlisting}[caption={실제 로그 발췌(hover\_bias, 10 Hz 요약)},label=lst:rate_log,breakindent=0pt,breakautoindent=false]
hover_bias fr+1 rl+3 fl+5 rr+6 mix[FR RL FL RR]= 1597 1599 1601 1602 est[r p y]= 0.00 -0.00 -0.00 gt[r p y]= 0.00 -0.00 -0.00 yaw[rate,cy]= -0.00,0.00 cp/cr(us)= 0.00/-0.00
hover_bias fr+1 rl+3 fl+5 rr+6 mix[FR RL FL RR]= 1598 1598 1602 1602 est[r p y]= 0.02 -0.04 -0.01 gt[r p y]= 0.02 -0.04 -0.01 yaw[rate,cy]= -0.00,0.00 cp/cr(us)= 0.66/-0.20
hover_bias fr+1 rl+3 fl+5 rr+6 mix[FR RL FL RR]= 1598 1598 1602 1601 est[r p y]= 0.05 -0.12 -0.03 gt[r p y]= 0.05 -0.12 -0.03 yaw[rate,cy]= -0.01,0.00 cp/cr(us)= 0.77/-0.25
\end{lstlisting}

\begin{table}[htb]
\centering
\caption{루프 주파수 달성(평균 및 변동 범위)}
\label{tab:loop_rates}
\begin{tabular}{lcc}
\toprule
루프 & 평균(Hz) & 변동(Hz) \\
\midrule
Inertial Measurement Unit (IMU) & 1000 & $\leq$ 5 \\
Barometer & 100 & $\leq$ 1 \\
PWM & 400 & $\leq$ 2 \\
\bottomrule
\end{tabular}
\end{table}


\begin{figure}[htb]
  \centering
  % \includegraphics[width=\columnwidth]{images/log_segments.png}
  \caption{실제 로그 발췌(hover\_bias, 10 Hz 요약). 자세한 스냅샷은 리스트 \ref{lst:hover_bias_log} 참조.}
  \label{fig:log_segments}
\end{figure}

\section{결론}
본 연구는 AirSim 내 공유메모리 기반 고주파 폐루프와 경량 안정화 패키지를 결합해, 시뮬레이터/개발 환경에서 안전하고 재현 가능하게 제어기를 벤치마크·튜닝할 수 있는 실용 프레임을 제시했다. 외부 프로그램(\emph{SIL controller})을 통해 IMU \(\sim\!1\,kHz\), Barometer \(\sim\!100\,Hz\), PWM \(400\,Hz\) 루프를 달성했고, 실제 로그로 이를 확인하였다.

\begin{itemize}
  \item \textbf{고주파 폐루프 달성}: in-process Shared Memory + SPSC로 지연/지터를 낮추고, \emph{SIL controller}에서 IMU \(\sim\!1\,kHz\), Barometer \(\sim\!100\,Hz\), PWM \(400\,Hz\)를 안정적으로 유지(리스트 \ref{lst:rate_log}).
  \item \textbf{스파이크/포화 감소, 자세 품질 개선}: D-term PT2$\times$2+클램프, Trim 램프-인, 추정기 게인 EMA 스무딩, EST/GT 연속 블렌딩, yaw rate-PI(anti-windup), 모터 Slew 제한의 결합으로 peak-to-peak와 포화 점유율을 유의하게 낮추고 yaw 정지(rate$\to$0), Attitude RMS를 개선.
  \item \textbf{재현 가능한 벤치마크}: 워밍업·윈도우·시나리오(S1--S4)와 공통 파라미터를 고정한 절차를 제시, 로그·표준 지표로 비교 가능성을 확보.
  \item \textbf{이식성}: 구성 요소를 FC 구조에 맞춰 모듈화하여, 시뮬레이터에서 검증한 동일 구조를 실제 FC로 최소 변경 이식 가능.
\end{itemize}

향후 과제로는 (i) 동적 노치 자동화와 추가 필터 체인 최적화, (ii) magnetometer/altitude(Baro) 통합 및 EKF 대체 비교, (iii) 다양한 부하/스케줄링 조건에서의 지연/지터 정량화, (iv) ROS2·HIL·실기체 포팅을 통한 외삽 검증, (v) 자동 튜닝/안정성 진단 도구화가 있다.

\begin{thebibliography}{12}
\bibitem{ardutune} ArduPilot, “Copter PID Tuning and Rate Controller (ATC\_RAT\_), Filter/Notch Docs,” 2025. \url{https://ardupilot.org/copter/docs/tuning.html}
\bibitem{px4tune} PX4 Dev Team, “Multicopter PID Tuning Guide,” 2025. \url{https://docs.px4.io/main/en/config_mc/pid_tuning_guide_multicopter.html}
\bibitem{astrom} K. J. \AA str\"{o}m and T. H\"{a}gglund, Advanced PID Control, ISA, 2006.
\bibitem{zaccarian} L. Zaccarian and A. R. Teel, Modern Anti-windup Synthesis, Princeton, 2011.
\bibitem{ardunotch} ArduPilot, “IMU Notch / Dynamic Notch Filtering,” 2025. \url{https://ardupilot.org/copter/docs/common-imu-notch-filtering.html}
\bibitem{ardufilter} ArduPilot, “IMU Filtering Reference (INS\_GYRO\_FILTER, PID D-term filtering),” 2025. \url{https://ardupilot.org/copter/docs/common-imu-filtering.html}
\bibitem{mahony} R. Mahony, T. Hamel, and J.-M. Pflimlin, “Nonlinear Complementary Filters on the Special Orthogonal Group,” IEEE TAC, 2008.
\bibitem{madgwick} S. O. H. Madgwick, “An Efficient Orientation Filter for Inertial and Inertial/Magnetic Sensor Arrays,” 2010.
\bibitem{barshalom} Y. Bar-Shalom, X.-R. Li, and T. Kirubarajan, Estimation with Applications to Tracking and Navigation, Wiley, 2001.
\bibitem{arduslew} ArduPilot, “MOT\_SLEWRATE: Motor Output Slew Rate,” 2025. \url{https://ardupilot.org/copter/docs/parameters.html#mot-slewrate-motor-output-slew-rate}
\bibitem{px4act} PX4 Dev Team, “Actuators / Output Configuration (output limiting, mixers),” 2025. \url{https://docs.px4.io/main/en/config/actuators.html}
\bibitem{airsim} S. Shah et al., “AirSim: High-fidelity visual and physical simulation for autonomous vehicles,” arXiv:1705.05065, 2017.
\end{thebibliography}

\end{document}


