AirSim 연동 고주파 제어 벤치마크와 안정화 키트: 설계·어블레이션·평가

AirSim-Integrated High-Frequency Control Benchmark and Stabilization Toolkit: Design, Ablation, and Evaluation

개요
  소형 멀티로터의 자세 제어는 모터 비대칭, 센서 바이어스, 추정기 게인 스위칭으로 인해 출력 스파이크(“팅김”)와 yaw 드리프트가 빈번하다. 본 연구는 실환경 제약을 반영한 경량 안정화 패키지를 제안한다. 제어 구조는 외부 자세 P-내부 각속도 PI+D의 2중 루프를 사용하며, (1) yaw rate-PI(anti-windup)로 영(0) 각속도 수렴, (2) 자이로 D-term에 2단 PT2 저역통과로 고주파 노이즈 억제, (3) 호버 트림을 3초 관측 후 선형 램프-인, (4) Mahony 추정기의 가감속 신뢰도 게인을 시간상수로 스무딩, (5) 추정기/GT 소스 전환을 연속 가중 블렌딩으로 계단 효과 제거, (6) 모터 Slew 제한으로 실제 출력 급변을 완화한다. AirSim-xsim 공유메모리 IPC를 통해 ~1 kHz 텔레메트리(타임스탬프 포함)와 400 Hz PWM의 고주파 폐루프를 구현하고, 링버퍼·비블로킹 소비·고해상도 동기화로 지연/지터를 최소화하였다. 모터/자이로 바이어스 및 동적 가속 조건에서 평가한 결과, 제안 기법은 스파이크 피크-투-피크와 포화 점유율을 유의하게 감소시키고, yaw 정지(rate→0)와 자세 RMS 오차를 개선하였다. 어블레이션으로 “한 번씩 튀는” 현상이 D-term 고주파 성분과 트림/게인 계단의 합성 효과임을 보였고, PT2+클램프 축소·트림 램프-인·게인 스무딩으로 제어입력 연속성을 회복함을 확인했다. 본 패키지는 오픈소스 오토파일럿의 실무 기법을 고주파 공유메모리 폐루프에 통합해, 시뮬레이터/HIL 개발에 즉시 적용 가능한 안정화 가이드를 제시한다.

Abstract
  Small multirotor attitude control often suffers from output spikes and yaw drift due to motor asymmetry, sensor bias, and estimator gain switching. This work proposes a lightweight stabilization package tailored to real-world constraints. The controller adopts a two-loop structure—outer attitude P and inner rate PI+D—and combines: (1) yaw rate-PI with anti-windup to drive angular rate to zero, (2) a two-stage PT2 low-pass filter on the gyro D-term to suppress high-frequency noise, (3) linear ramp-in of hover trim after a 3 s observation window, (4) time-constant-based smoothing of the Mahony estimator’s acceleration-trust gains, (5) continuous estimator/ground-truth source blending to eliminate step transitions, and (6) motor slew limiting to reduce abrupt output changes. We implement a high-frequency closed loop via AirSim-xsim shared-memory IPC, consuming timestamped telemetry at ~1 kHz and transmitting PWM at 400 Hz; a ring buffer, non-blocking consumption, and high-resolution timebase synchronization minimize latency and jitter. Under injected motor/gyro biases and dynamic acceleration, the proposed methods significantly reduce peak-to-peak spikes and saturation occupancy, achieve yaw stop (rate → 0), and improve attitude RMS error. Ablation shows that the intermittent “single-pop” spikes arise from the joint effects of high-frequency D-term content and stepwise trim/gain application; continuity is restored via PT2 with tighter clamps, trim ramp-in, and gain smoothing. By integrating filtering, anti-windup, and smoothing practices common in open-source autopilots into a high-frequency shared-memory loop, this work offers a practical stabilization guide readily applicable to simulator and HIL development.

keyword: 멀티로터, 자세 제어, D-term 필터링, anti-windup, 추정기 스무딩, 공유메모리 IPC, AirSim

Ⅰ. 서 론
  소형 멀티로터의 자세 제어는 모터/기체 비대칭, 센서 바이어스와 고주파 노이즈, 추정기 게인의 상황별 스위칭 등 실환경 요인으로 인해 출력 스파이크(“팅김”)와 장기적 yaw 드리프트가 빈번하게 발생한다. 이러한 현상은 D-term이 자이로 고주파 성분에 과민하고, 트림(바이어스 보정)이나 추정기 게인이 계단(step)으로 적용되며, 액추에이터의 물리적 제약(포화·율제한)과 상호작용하면서 증폭된다. 결과적으로 단일 파라미터 튜닝이나 개별 기법만으로는 재현성과 강인성을 동시에 확보하기 어렵다.
본 연구는 AirSim과 제어기 사이에 공유 메모리 기반 IPC(Inter-Process Communication; XSimIo)를 사용해 약 1 kHz 텔레메트리(타임스탬프 포함)와 400 Hz PWM으로 구성된 고주파 폐루프를 구현하고, 지연·지터를 최소화한 환경에서 “튀는” 현상과 yaw 드리프트를 체계적으로 완화하는 경량 안정화 패키지를 제안한다. 제어 구조는 외부 자세 P-내부 각속도 PI+D의 표준 2중 루프를 채택하되, 필터링·스무딩·anti-windup·출력 제약을 충돌 없이 결합하여 제어입력의 연속성과 고주파 강인성을 동시에 달성하도록 설계하였다.

본 연구의 기여는 다음과 같다.
- 실용 안정화 패키지. yaw rate-PI(anti-windup)로 영(0) 각속도 수렴을 보장하고, 자이로 D-term에 2단 PT2(2차 저역통과)와 타이트한 클램프(±12 μs)를 적용해 고주파 스파이크를 억제한다.
- 연속성 확보. 호버 트림을 3 s 관측 후 0.8 s 선형 램프-인으로 적용하고, Mahony 추정기의 가감속 신뢰도 게인을 시간상수 기반 EMA(Exponential Moving Average)로 스무딩하며, 추정기/GT(Ground Truth) 소스를 연속 가중 블렌딩으로 전환해 단계 입력을 제거한다.
- 액추에이터 보호. 모터 Slew(명령 변화율) 제한(샘플 간 4 μs)으로 실제 출력 급변을 완화하고 포화 점유율을 저감한다.
- 고주파 벤치마크. AirSim-xsim 공유메모리 폐루프에서 모터/자이로 바이어스·동적 가속 조건을 가하며 cp/cr, cy, yaw rate, 포화율, peak-to-peak 등 지표로 정량 평가한다.
- 원인 분해/어블레이션. “한 번씩 튀는” 현상이 D-term 고주파 성분과 트림/게인 계단의 합성 효과임을 보이고, PT2+클램프 축소·트림 램프-인·게인 스무딩의 결합이 제어입력 연속성을 회복함을 실증한다(섹션 V 참조).
 이로써 제안 패키지는 스파이크의 피크-투-피크와 포화 점유율을 유의하게 감소시키고, yaw 정지(rate→0)와 자세 RMS 오차를 개선한다. 본 접근은 오픈소스 오토파일럿(예: ArduPilot)에서 축적된 실무 원리를 고주파 공유메모리 폐루프 맥락에 맞춰 통합·스무딩·램핑한 점에서 실용적 독창성을 가지며, 시뮬레이터 및 HIL 기반 개발에 즉시 적용 가능한 안정화 레퍼런스를 제공한다.

Ⅱ. 관련 연구 (Related Works)
오픈소스 오토파일럿의 표준 구조와 실무 튜닝
ArduPilot/PX4는 외부 자세 P-내부 각속도 PID(혹은 PI+D) 2중 루프를 표준으로 채택하며, 실기체 진동·노이즈·포화 대응을 위해 다음 실무 기능을 제공한다: D-term 저역통과(PT1/PT2) 및 노치 필터, 적분 제한/프리즈(anti-windup), 모터 출력 Slew 제한, Feedforward, 센서 필터 체인(자이로/가속도) 등. 이들 기능의 개별 효과와 튜닝 지침은 문서·가이드·포럼에서 광범위하게 축적되어 있다. 다만 “여러 기법을 동시에 조합했을 때의 상호작용”을 어블레이션으로 체계적으로 정리한 공개 자료는 제한적이다. [1], [2]
PID 적분기 관리: Anti-windup과 Bumpless 전이
포화·율제한이 있는 액추에이터에서 적분기 과적분은 오버슛/긴 복구시간을 유발한다. 이를 완화하기 위해 고전 제어 문헌에서는 적분 제한(Integral Limit), 백-계산(Back-calculation), 조건부 적분(Conditional Integration), I-freeze/decay(포화 시 적분 동결/감쇠), 목표/게인 변경 시의 Bumpless Transfer(출력 계단 최소화) 등을 제안한다. 멀티로터 분야에서도 유사 원칙이 적용되며, 속도 루프의 steady-state 바이어스를 제거하기 위해 rate-PI + anti-windup이 널리 사용된다. [3], [4]
D-term 필터링과 미분 성형
미분(D)은 고주파 노이즈·진동 모드에 민감하기 때문에, PT1/PT2 저역통과, 정적/동적 노치(dyn-notch), 미분 성형(differentiator shaping)으로 고주파 성분을 억제한다. 실무에서는 D-term에 독립 필터 체인을 두고, D 클램프(출력 제한)와 함께 모터 Slew 제한을 병행해 “한 번씩 튀는 스파이크”와 포화 점유율을 줄인다. [5], [6]
추정기 게인 스케줄링과 스무딩
Mahony/Madgwick/EKF 류 추정기는 동적 상태(가속도 크기, 기동률, 진동 환경)에 따라 가속도 신뢰도(kp/ki)를 스케줄링한다. 그러나 구간별 “계단 전환”은 제어 루프에 단계 입력을 유입해 과도 응답을 키울 수 있어, 시간상수(EMA) 기반의 연속 스무딩으로 천이를 부드럽게 하는 접근이 보고된다. [7], [8]
소스 전환(EST/GT)과 연속 블렌딩
센서/소스 전환(예: EST↔GT, 센서 페일오버)은 단계 전이가 발생하면 루프에 큰 버스트를 주기 쉽다. 이를 막기 위해 일정 기간 동안 연속 가중(0→1) 블렌딩을 적용, 전환을 연속화하는 기법이 센서퓨전·로버스트 필터링 문헌에 제안되어 왔다. [9]
액추에이터 제약: 포화·율제한과 Slew
액추에이터 포화/율제한(rate limit)은 anti-windup 설계의 핵심 전제이며, 실제 시스템에서는 출력 Slew(명령 변화율 제한)로 전기/기계 과도를 억제하고 구동부 보호와 응답 연속성을 확보한다. 멀티로터 제어기에서도 per-모터 Slew 설정이 일반화되어 있다. [10], [11]
시뮬레이션/HIL 검증 관행
AirSim, Gazebo 등 시뮬레이터와 HIL 환경에서 PID/필터/튜닝 연구가 다수 보고되었다. 다만 AirSim과 컨트롤러를 “공유 메모리 IPC”로 직결해 ~1 kHz 텔레메트리(타임스탬프 포함)와 400 Hz PWM을 고주파 폐루프로 운용하며, 트림 램프-인·EST/GT 블렌딩·PT2 D+클램프+Slew를 “동시에” 적용하고 어블레이션/정량 지표(peak-to-peak, 포화율, yaw rate, cp/cr, cy)로 효과를 체계 비교한 공개 사례는 제한적이다. [12]
본 연구와의 차별성(정리)
“개별 기법”의 도입이 아니라, 고주파 공유메모리 폐루프 맥락에서 트림 램프-인, EST/GT 연속 블렌딩, PT2 D(타이트 클램프), 모터 Slew, yaw rate-PI(anti-windup)를 충돌 없이 “동시에” 결합.
“튀는 스파이크”의 원인을 D 고주파 성분 + 트림/게인 계단 적용의 합성효과로 분해하고, 세부 기법(PT2+D클램프·트림 램프-인·게인 스무딩)이 제어입력 연속성을 회복한다는 점을 어블레이션과 정량 지표로 증명.
AirSim-xsim 공유메모리 IPC로 실험의 지연/지터를 낮춘 상태에서 재현 가능한 벤치마크 프로토콜을 제시.

Reference.
[1] ArduPilot, “Copter PID Tuning and Rate Controller (ATC_RAT_), Filter/Notch Docs.” Available: https://ardupilot.org/copter/docs/tuning.html (accessed Oct. 2025).
[2] PX4 Dev Team, “Multicopter PID Tuning Guide (MC_* parameters, D-term cutoff).” Available: https://docs.px4.io/main/en/config_mc/pid_tuning_guide_multicopter.html (accessed Oct. 2025).
[3] K. J. Åström and T. Hägglund, Advanced PID Control. Research Triangle Park, NC: ISA, 2006.
[4] L. Zaccarian and A. R. Teel, Modern Anti-windup Synthesis: Control Augmentation for Actuator Saturation. Princeton, NJ: Princeton Univ. Press, 2011.
[5] ArduPilot, “IMU Notch / Dynamic Notch Filtering.” Available: https://ardupilot.org/copter/docs/common-imu-notch-filtering.html (accessed Oct. 2025).
[6] ArduPilot, “IMU Filtering Reference (INS_GYRO_FILTER, PID D-term filtering).” Available: https://ardupilot.org/copter/docs/common-imu-filtering.html (accessed Oct. 2025).
[7] R. Mahony, T. Hamel, and J.-M. Pflimlin, “Nonlinear complementary filters on the special orthogonal group,” Proc. 47th IEEE CDC, 2008, pp. 1477-1484. doi:10.1109/CDC.2008.4739011.
[8] S. O. H. Madgwick, “An efficient orientation filter for inertial/magnetic sensor arrays,” Tech. Report, 2010. Available: http://x-io.co.uk/open-source-imu-and-ahrs-algorithms/ (accessed Oct. 2025).
[9] Y. Bar-Shalom, X.-R. Li, and T. Kirubarajan, Estimation with Applications to Tracking and Navigation. Hoboken, NJ: Wiley, 2001.
[10] ArduPilot, “MOT_SLEWRATE: Motor Output Slew Rate.” Available: https://ardupilot.org/copter/docs/parameters.html#mot-slewrate-motor-output-slew-rate (accessed Oct. 2025).
[11] PX4 Dev Team, “Actuators / Output Configuration (output limiting, mixers).” Available: https://docs.px4.io/main/en/config/actuators.html (accessed Oct. 2025).
[12] S. Shah et al., “AirSim: High-fidelity visual and physical simulation for autonomous vehicles,” arXiv:1705.05065, 2017.

III. 방법/시스템 구현(methods/system)
    [다이어그램1 - AirSim ↔ Virtual FC 구조: Mahony 자세추정기(Adaptive gains, EMA), Attitude P → Rate PI+D(D: PT2×2, clamp), Controller(400 Hz), EST/GT 블렌딩, Trim ramp-in]
    그림 1. AirSim-xsim-Virtual FC 블록도. 텔레메트리(~1 kHz)와 PWM(400 Hz) 경로, Mahony 자세추정기(Adaptive gains, EMA), Attitude P → Rate PI+D(D: PT2×2, clamp ±12 μs), EST/GT 블렌딩(τ≈0.25 s), 트림 램프-인(3 s→0.8 s, ±50 μs).

 시스템 개요
  AirSim-xsim 공유메모리(IPC)로 컨트롤러를 연결해 타임스탬프가 포함된 텔레메트리를 약 1 kHz로 소비하고, PWM 명령을 400 Hz로 송신한다. 단일생산자-단일소비자(SPSC) 링버퍼와 비블로킹 소비, zero-copy 경로, 고해상도 시계 동기화를 통해 센서→제어 지연과 주기 지터를 줄였다. 추정/제어 스레드의 우선순위를 상향해 고주파 폐루프의 안정성을 확보했다. 기호: q(자세 쿼터니언), ex/ey(소각오차), wx_ref/wy_ref(각속도 참조). Virtual FC의 Sensors는 ACC, GYRO만 사용(추정기 관점; MAG/BARO 미사용).

 상태추정기
  Mahony 보완필터 기반으로 쿼터니언 자세 q를 추정한다. 각 샘플의 타임스탬프로 가변 Δt를 계산해 적분하며, 매 스텝 q를 정규화한다. 가속도 신뢰도는 크기 편차 |‖a‖−g|가 작을 때만 보정에 사용하고, kp·ki는 ∥a∥ 근접도에 따라 명령치를 정한 뒤 시간상수(τ) 기반 EMA로 스무딩해 게인 계단을 방지한다. 정지/저속 구간에서는 자이로 바이어스를 저속 EMA로 온라인 추정해 장기 드리프트를 억제한다. (상세 임계·상수는 IV 절 참조)

 제어기(Attitude P → Rate PI+D, yaw rate-PI)
  외부 루프는 쿼터니언 소각오차(ex, ey)에 데드존을 적용한 뒤 각속도 참조(wx_ref, wy_ref)를 만든다. 내부 루프는 각속도 PI에 조건부 적분·적분 한계(anti-windup)를 적용한다. D-term은 바이어스 보정 자이로율을 입력으로 2단 PT2 저역통과를 거친 후 타이트 클램프를 적용해 스파이크를 억제한다. yaw 축은 angle-P=0의 rate-PI(필요 시 소량 D 병행)로 0(rad/s) 수렴을 우선한다. (세부 한계값은 IV 절)

 트림(hover bias) 램프-인
  이륙 초기 관측 윈도우에서 ex/ey 평균 바이어스를 추정해 각도→출력 변환 후 한계로 포화한 목표 트림을 산출한다. 적용은 선형 램프-인으로 수행해 계단 적용에 따른 과도 킥을 방지한다. 트림은 믹서 이전 단계에서 롤/피치 보정에 더해지며, 램프-인 완료 후 고정된다. (관측 시간, 램프-인 시간, 한계 및 변환계수는 IV 절)

 EST/GT 소스 블렌딩
  추정(EST)과 GT 소스 전환은 이산 스위치 대신 연속 가중(gt_blend)으로 처리한다. 안전/품질 조건을 만족하면 GT 가중을 상승시키고, 해제 시 1차 EMA로 연속 복귀시켜 단계 전이를 제거한다. 최종 제어 오차는 ex=(1−gt_blend)·ex_est+gt_blend·ex_gt, ey도 동일하게 합성한다. (임계와 시간상수는 IV 절)
  
IV. 실험 설정(experimental setup)
환경(Environment)
AirSim-xsim 공유메모리 IPC로 컨트롤러와 시뮬레이터를 연결한다. 텔레메트리는 ~1 kHz(타임스탬프 포함) 비블로킹 소비, PWM 명령은 400 Hz 송신으로 고주파 폐루프를 구성한다. 고해상도 시계 동기화와 링버퍼를 사용해 지연·지터를 최소화한다.
소프트웨어/플랫폼
Windows 10, AirSim(동일 월드/기체 파라미터 고정), 컨트롤러는 XSimIo 기반 공유메모리 IPC 구현을 사용한다. 스레드 우선순위를 상향하고 타이머 분해능을 1 ms로 설정한다.
테스트 케이스(Scenarios)
S1-Hover: 이륙(1800 μs, 7 s) 후 호버(1600 μs)에서 바이어스 관측 및 제어(기본 바이어스 예: FR+1, RL+3, FL+5, RR+6 μs).S2-Gyro Bias: z-rate에 상수 바이어스(예: ±0.02-0.05 rad/s) 주입.S3-Dynamic Accel: 사인/스텝 외란(0.3-0.8 g, 0.5-2 Hz) 적용.S4-Aggressive Tilt: 빠른 롤/피치 스텝 참조로 과도 응답 확인.
비교군(Configurations)
Baseline: D-term PT1(또는 단일 LPF), 트림 즉시 적용(램프-인 없음), 게인 계단 전환, Slew 8 μs, yaw D 중심.Proposed: D-term PT2(2단) + 타이트 클램프(±12 μs), 트림 3 s 관측→0.8 s 램프-인, Mahony 게인 스무딩(τ≈0.35 s), EST/GT 연속 블렌딩(τ≈0.25 s), Slew 4 μs, yaw rate-PI(anti-windup).
실험 프로토콜(Protocol)
각 시나리오당 N=20 반복, 초기 높이·자세·시드 고정. 이륙 구간 제외 후 분석 윈도우(예: 30-120 s)를 사용한다. Warm-up 5 s를 버리고 통계 산출한다.
지표 정의(Metrics)
Peak-to-Peak(μs): 모터 명령의 최대-최소.포화율(%): [≤1002, ≥1998] μs 구간 점유 시간 비율.Attitude RMS/95th(deg): 롤/피치 오차.Steady Yaw-Rate(rad/s): |ωz|의 평균 및 95th.cp/cr(μs): pitch/roll 보정 크기(절대값 평균).cy(μs): yaw 제어 출력 평균 및 분산.지연/지터(ms): 텔레메트리 Δt, PWM 주기 히스토그램/박스플롯.
파라미터(고정/튜닝)
제어: ANG_P_R/P, roll/pitch(Kp, Ki, I-limit), yaw(Ki, I-limit, Kd), deadzone.필터: PT2 알파(2단), D-클램프(±12 μs).램프/블렌딩: 트림 관측 3 s·램프-인 0.8 s·한계 ±50 μs, GT 게이팅 임계 5°/5°, 블렌딩 τ≈0.25 s.Slew: 4 μs/스텝(민감도 테스트 시 6-8 μs로 스윕).
통계 처리(Statistics)
평균±표준편차와 95% CI를 보고한다. Baseline vs Proposed는 페어 비교(필요 시 비모수 검정)로 유의성 확인, 효과크기(Cohen’s d) 병기.
데이터 수집/로깅(Logging)
요약(10 Hz): 모터 명령, est/gt Euler(r,p,y), yaw[rate, cy], cp/cr, 포화 플래그.진단(옵션 50 Hz): est vs gt 쿼터니언 오차, ex/ey(est/gt), gt_blend 가중, P/I/D 분해, 믹스 평균.
안전/게이팅(Safety/Gating)
est-gt 오차>5° 또는 큰 롤/피치 시 GT 가중을 일시 상승, 해제 시 블렌딩으로 천이. 포화 지속 시 적분 리셋/프리즈와 Slew 우선 적용.
재현성(Reproducibility)
코드 커밋 해시, 파라미터 파일, 실행 스크립트, 로그 경로/명명 규칙을 명시한다. 실험 스윕(시나리오×구성×반복) 자동화 스크립트를 제공한다.

% 표 1: 핵심 파라미터 (LaTeX 표)
\begin{table}[htb]
\centering
\caption{핵심 파라미터(예시) — 상세 근거는 IV 절 참조}
\begin{tabular}{ll}
\hline
항목 & 값 \\
\hline
EST/GT 임계 & 5$^\circ$ (쿼터니언 오차 또는 $|$roll$|$/$|$pitch$|$) \\
블렌딩(EMA) & $\tau\!\approx\!0.25$ s, $\alpha_{\max}=0.25$ \\
Attitude 데드존 & 0.005 rad \\
D-클램프 & $\pm 12\,\mu s$ (gyro D-term) \\
Trim(관측/램프/한계) & 3 s / 0.8 s / $\pm 50\,\mu s$ \\
각도→출력 계수 & 80 $\mu s/\mathrm{rad}$ \\
Slew 제한 & 4 $\mu s$/step \\
\hline
\end{tabular}
\label{tab:key_params}
\end{table}

% 표 2: 섹션 IV 수치 교차참조 (LaTeX 표)
\begin{table}[htb]
\centering
\caption{섹션 IV 수치 교차참조(사용 위치)}
\begin{tabular}{ll}
\hline
수치 & 사용 위치 \\
\hline
5$^\circ$ 임계 & III-EST/GT 블렌딩; IV-파라미터/안전 \\
$\tau\!\approx\!0.25$ s & III-EST/GT 블렌딩; IV-파라미터 \\
0.005 rad 데드존 & III-제어기; IV-파라미터 \\
$\pm 12\,\mu s$ D-클램프 & III-제어기; IV-파라미터 \\
3 s / 0.8 s / $\pm 50\,\mu s$ & III-트림; IV-파라미터 \\
80 $\mu s/\mathrm{rad}$ & III-트림; IV-파라미터 \\
4 $\mu s$/step Slew & III-제어기/보호; IV-파라미터 \\
\hline
\end{tabular}
\label{tab:cross_ref}
\end{table}

V. 결과 및 어블레이션(results & ablations)
VI. 토의(discussion: 한계/의미/실무가치)
VII. 결론(conclusion)
참고문헌
부록/재현성: Appendix A/B로 표기 권장 (번호 대신 Appendix)

